
%============================= HEADER =================================
\chapter{Introduzione}\label{sec:i}

In un suo articolo del 1952, Biermann \cite{1952LIACo...4..251B} si propone di analizzare le anomalie osservate nel moto delle code delle comete, con l'intenzione di applicare i risultati della \emph{fisica dei plasmi} in ambito astrofisico.
\par La materia ionizzata che costituisce le code delle comete presenta spesso accelerazioni dell'ordine di $100 \, g_{\odot}$, con $g_{\odot}$ l'accelerazione di gravità solare superficiale. 
\par \emph{Rebus sic stantibus}, è impossibile attribuire un'accelerazione di tale entità alla pressione di radiazione solare, che porterebbe ad una forza di oscillazione per la transizione $\gtrapprox 20$, ovvero oltre due ordini di grandezza rispetto a quanto osservato.
\par Riprendendo l'analisi effettuata da Unsöld e Chapman \cite{1949Obs....69..219U}, basata su osservazioni spettrografiche e nel radio, Biermann assume una densità massima alla radiazione corpuscolare del Sole di $10^{5}\, \text{ioni}/\text{cm}^{3}$ per tempeste geomagnetiche molto intense, mentre in condizioni "normali" e magneticamente indisturbate, la densità di particelle di radiazione corpuscolare è di circa $10^{3}\, \text{ioni}/\text{cm}^{3}$.
\par Le accelerazioni osservate sono compatibili con l'assunzione di un moto influenzato non solo dalla radiazione elettromagnetica prodotta dalla materia ionizzata, ma anche dalle collisioni tra le particelle che costituiscono la radiazione corpuscolare (elettroni e protoni) e le molecole ionizzate che compongono la coda stessa (e.g. $\text{CO}^{+}, \, \text{N}_{2}^{+}$ e altri ioni). 
\par Per completezza, riporto a titolo di esempio l'equazione del bilancio dei momenti per una delle tre componenti coinvolte\footnote{Per una trattazione completa, rimando all'articolo originale di Biermann \cite{1952LIACo...4..251B}\,.}:
\begin{equation}
	m_{m}\frac{d\textbf{v}_{m}}{dt} + \mu_{mp}\gamma_{mp}(\textbf{v}_{m}-\textbf{v}_{p}) + \mu_{me}\gamma_{me}(\textbf{v}_{m}-\textbf{v}_{e}) = e\textbf{E} + \frac{e}{c}[\textbf{v}_{m}\textbf{H}]
	\label{eq:biermann}
\end{equation}
dove $m_{m}$ è la massa della molecola, $m_p$ e $m_{e}$ rispettivamente massa di protone ed elettrone, $\textbf{v}_{m}$ la velocità delle molecole della coda, $\mu_{ij}$ (con $i,\,j$ tra $e,\,p,\,m$) la massa ridotta della coppia di particelle $i$, $j$ e $\gamma_{ij}$ la probabilità che una particolare componente $i$ collida con la componente $j$ per unità di tempo. \textbf{E}, \textbf{H}, $e$ e $c$ hanno il significato usuale. 
\par Ammettendo alcune semplificazioni, il sistema di tre equazioni può essere risolto. Notabilmente, assumendo densità numeriche di particelle dell'ordine di quelle stimate da Unsöld e Chapman, e una velocità per i protoni di $v_{p} \simeq 10^{8}\, \text{cm}/\text{s}$, si ottengono accelerazioni per gli ioni $\text{CO}^{+}, \, \text{N}_{2}^{+}$ in un intervallo compreso tra $10^{2}-10^{4}\, \text{cm}/\text{s}^{2}$.
\par Se ne deduce che le accelerazioni osservate per tali ioni nelle code possono essere consistentemente spiegate in termini di collisioni tra particelle della radiazione corpuscolare e molecole ionizzate. 
\par Nelle sezioni a seguire, ci proponiamo sviluppare un modello di \emph{vento solare} che possa essere realisticamente responsabile per l'eiezione di $10^{14}-10^{15}\,\text{g}$ di idrogeno ionizzato al secondo, con velocità di $1000\, \text{km}/\text{s}$, come si richiede per spiegare le osservazioni di Biermann. Nel fare ciò, adotteremo l'approccio seguito da Parker \cite{1958ApJ...128..664P}.

