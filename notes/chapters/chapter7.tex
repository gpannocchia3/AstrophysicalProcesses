\chapter{Accretion in binary systems} \label{ch:vii}
\section{Introduction}
Binary systems of stars (or black holes, as we'll see in the next chapter) are one of the most common configurations were we take a good look at the sky (that is, assuming we're able to indeed resolve the binary system).
\par On top of that, differently from their single counterpart, binaries reveal more about themselves, notably their masses and dimensions, than do other astronomical objects; this is particularly true in the case of \emph{eclipsing binaries}.
\par It may come as no surprise hearing that angular momentum is going to play a major role in the study of accretion in binary systems. Fo example, in many cases, the transferred material cannot land on the accreting star until it has rid itself of most of its angular momentum, leading to the formation of \emph{accretion disks}, which we'll be the focus of quite a few sections among the following.
\section{Interacting binaries}
As always, I'm referring to \cite{Frank_King_Raine_2002} for this part.
\par There are two main reasons many binaries transfer matter at some stage of their
evolutionary lifetimes:
\begin{itemize}
    \item (i) one of the stars in a binary system may increase in radius (for example because it's entering the \emph{giant phase} of its evolution), or the binary separation shrink (this will be treated more carefully in the last chapter of this notes), to the point where the gravitational pull of the companion can remove the outer layers of its envelope;
    \item (ii) one of the stars may eject much of its mass in the form of a stellar wind (again, perhaps because its a particularly massive giant star); some of this material will be captured gravitationally by the companion.
\end{itemize}
We'll focus on the (i) scenario.
\section{Roche lobes}
We shall now try to understand what is it of a small test particle living in the combined gravitational potential of the two stars that make the binary, orbiting each other under the influence of their mutual gravitational attractions. The two bodies of the binary are assumed to be so massive (in respect to the test particle) that the test particle has no influence over their orbits.
\par Assuming this to hold, the system is pretty much reduced to a Physics 1 problem of two bodies interacting with a radial potential. So we already know the solution. The two stars execute Kepler orbits about each other in a plane.
\par As a further simplification, we'll assume these orbits to be circular. This is usually a good approximation for binary systems, since tidal effects tend to circularize originally eccentric orbits on timescales short compared to the time over which mass transfer occurs.
\input{chapters/binary}
\par Consider a binary system as crudely depicted in Fig.\ref{fig:binaries}. For the purposes of this section, we're going to assume that the two stars of the binary can be assumed to be point-like sources of potential with separation $a$.
\par That being said, the binary period\footnote{Here we're using the angular frequency rather than the period. It's the same up to some power of $2\pi$} is inferred from Kepler's third law
\begin{equation}
    \Omega^{2} = \frac{G(m_{1}+m_{2})}{a^{2}}
    \label{eq:kepler_3}
\end{equation}
A more aesthetically pleasing version of Fig.\ref{fig:binaries} is given below.
\begin{figure}[h!]
    \includegraphics[width=0.8\textwidth]{img/binary_pleasing.png}
    \caption{A binary system of two stars orbiting around their common center of mass. Credits: Frank, King and Raine.}
    \label{fig:bin_aesth}
\end{figure}
\par In the corotating frame, the potential generated from the system is 
\begin{equation*}
    \phi(\vec{r}\,) = -\frac{Gm_{1}}{|\vec{r}-\vec{r}_{1}|} - \frac{Gm_{2}}{|\vec{r}-\vec{r}_{2}|} -\frac{1}{2}(\vec{\Omega}\wedge\vec{r}\,)^{2}
\end{equation*}
where $\vec{r_{1}}$, $\vec{r_{2}}$ are the position vectors of the centres of the two stars.
\par It's clear that any gas flow between the two stars we should be able to describe, in principle, through Euler's equation, which will be of the form
\begin{equation*}
    \frac{\partial \vec{v}}{\partial t} + (\vec{v}\cdot\nabla)\cdot \vec{v} = -\frac{\nabla p}{\rho}-\nabla\phi -2\vec{\Omega}\wedge\vec{v}
\end{equation*}
\par Considerable insight about accretion problems can be gained if we consider the equipotential surfaces of the the gravitational potential, and in particular their section on the orbital plane.
\begin{figure}[h!]
    \centering
    \includegraphics[width=0.7\textwidth]{img/roche.png}
    \caption{Sections in the orbital plane of the Roche potential $\phi$. The five Lagrange point where the potential is zero are marked in the picture with a $L_{i}$. The saddle point $L_{1}$ is said "inner Lagrange point" and forms a "pass" from the two Roche lobes (\textbf{3}). Credits: Frank, King and Raine.}
    \label{fig:roche}
\end{figure}
\par What if one of the two stars external layer bloates up and steps out of its Roche lobe? Since the lobes are essentially demarcation lines between the two (single star) potentials, once a Roche lobe is overflown, material will fall on the other object, captured by the gravitational potential of the other star.
\par This type of mass transfer is called \emph{Roche lobe overflow}. Since mass is removed from the star quite readily when the lobe is filled, the star cannot be significantly larger than its Roche lobe. The star losing mass is typically named \emph{donor}, while the one gaining mass is called \emph{accretor}.
\par Generally, there's a non-zero pressure gradient near the $L_{1}$ Lagrange point, with material often flowing at supersonic speeds\footnote{For those who may be interested in the subject, I do suggest reading into something of I. S. Shklovskii, for example \cite{1972SvA....15..886S} or \href{https://ui.adsabs.harvard.edu}{here}.}. Clearly, if we introduce angular momentum back into the picture, Coriolis' forces will arise, deviating the trajectory of said gas until it settles into a ring.
\begin{figure}[h!]
    \centering
    \includegraphics[width=0.8\textwidth]{img/disk_form.png}
    \caption{Perspective view of the spiraling gas. Credits: Frank, King and Raine.}
    \label{fig:spiral_gas}
\end{figure}
\par To a good approximation we can take the stream trajectory as the orbit of a test particle released from rest at $L_{1}$, and thus with a given angular momentum, falling in the gravitational field of the accreter alone. This would give an elliptical orbit lying in the binary plane: the presence of the donor causes this to precess slowly. A continuous stream trying to follow this orbit will therefore intersect itself, resulting in dissipation of energy via shocks.
\par On the other hand, the gas has little opportunity to rid itself of the angular momentum it had on leaving $L_{1}$, so it will tend to the orbit of lowest energy for a given angular momentum, a circular orbit. We thus expect the gas initially to orbit the primary in the binary plane at a radius $R_{c}$ such that the Kepler orbit at $R_{c}$ has the same specific angular momentum as the transferring gas had on passing through $L_{1}$.
\par If we consider Euler's equation for the vorticity in presence of viscosity
\begin{equation*}
    \frac{\text{D}\vec{\omega}}{\text{D}t} = \nu \nabla^{2}\vec{\omega}
\end{equation*}
we discover that a sensible solution may be of the form 
\begin{equation*}
    \omega \propto \exp\left(-\frac{r^{2}}{2\nu\tau}\right)
\end{equation*}
so that material diffuses all over until eventually forms a \emph{disk}, which will be the focus of the next section.
\subsection*{AGNs and BEBs (Big Energetic Bros)}
Jokes aside, as we'll have the pleasure to discuss in the next few sections, accretion can't be possibly limited only to "small" objects with hard boundaries\footnote{The notion of "hard" for something made of hot plasma may be somewhat deceptive, but we'll stick with the notation.} like stars.
\par As far as our understanding goes, the very things at the center of galaxies are constantly accreting the nearby matter, thus depleting gravitational energy in (often) spectacular forms which we're able to detect! On top of that, we're aware of the existence of binary systems made of a "small" black holes and something else; this is the case, for example of the Sco-X1 binary system.
\par AGNs (\emph{Active Galactic Nucleus}) are particularly relevant in the picture of accretion since they've been observed to be able to accrete both spherically (no angular momentum) and non-spherically (non-zero angular momentum), with the formation of a disk. Our common understanding of AGNs relies on the (fairly tested) belif that they're powered by supermassive black holes in the range of $10^{5}-10^{8}\,M_{\odot}$.
\par To give the world the illusion that we're really understanding what's going on in there, AGNs have been given a nice classification based on the type of galaxy they're found in and on their spectral features (when needed):
\begin{itemize}
    \item Blazers $\to$ Elliptical galaxies
    \item Type I Seyfert $\to$ (Late type) Spiral Galaxies with narrow absorption lines
    \item Type II Seyfert $\to$ (Late type) Spiral Galaxies with broad absorption lines\footnote{The broadness or narrowness of the lines is related to their velocity in respect to us. Narrow absorption lines mean that the AGN is far from us, while broad absorption lines are typical of closer AGNs.}
    \item Radio Loud Quasars $\to$ (Early type) Elliptical galaxies
    \item Radio Quiet Quasars $\to$ (Early type) Elliptical galaxies
\end{itemize}
\par It may be worth pointing out that this classification is pretty much a hand-wavingly way to present the same object in different scenarios. In fact, all points in the previous list represent pretty much the same objects, with the relative tilt from the observation direction the only thing that is really changing. But this should not surprise you. Relativistic beaming is heavily frequency-dependent and direction-dependent, so depending on the direction you're looking from, you're going to have really different results.
\par As you can easily cross-check in your favorite book of (advanced) electrodynamics \cite{Jackson:100964}, the angular distribution of radiation emitted by an accelerated charge can be calculated starting from the relativistic-adjusted Poynting's vector.
\par Doing this, you'll find out that when the velocity and the acceleration of the emitting particle are parallel, the emitted power per solid angle is given by 
\begin{equation*}
    \frac{\text{d}P}{\text{d}\Omega} = \frac{e^{2}\dot{v}^{2}}{4\pi c^{3}}\frac{\sin^{2}\theta}{(1-\beta\cos\theta)^{5}}
\end{equation*}
where $\theta$ is the angle of observation measured from the direction of $\vec{\beta}$. From this expression it's clear the heavy direction-dependence; to find the frequency spectrum you need to put in some more work.
\par Although not entirely nor strictly related to this, I think you can get an idea of what's going on by watching this short clip by \href{https://www.youtube.com/shorts/RCSNL0lop8I}{Kip Thorne}.
\par Another interesting fact is that most of these systems present powerful jets which may remain coherent for \emph{kiloparsecs}.
\par How? Now, that's a good question. The most accredited theory has been developed by R. D. Blandford and R. L. Znajek in 1977 \cite{1977MNRAS.179..433B}; we won't go much deeper than this here\footnote{To quote a book I've studied from: "The author of this notes does not claim to have any particular insight on the subject."} but they have essentially speculated the formation of magnetic funnels around spinning black holes.
\par When you start considering spinning black holes a whole other handful of problems arises. For once, you can no longer use the nice and simple Schwarzschild metric to describe the BH, but you have to use Kerr's. Even then, it has been shown that in presence of Kerr's black holes, if the accretion disk is (somehow) tilted in respect to the rotation axis of the BH, an effect called \emph{Bardeen precession} arises \cite{1975ApJ...195L..65B}, leading to a considerable part of the radiation emitted in the central part of the disk to be reabsorbed, inducing observable changes in the X-ray spectrum.
\par And then you have curious fellas like \emph{obscure black holes}, which are essentially geometrically thick (we'll see more about this in the next section) tori through which only infrared radiation may pass. 
\section{Disk formation}
If we ignore for the time being viscosity in Euler's equation, we may write
\begin{equation*}
    \partial_{t}\vec{v} + (\vec{v}\cdot \nabla)\cdot \vec{v} = -\frac{\nabla p}{\rho} -\frac{GM}{r^{2}}\hat{r}
\end{equation*}
Assuming ideal gas behavior
\begin{equation*}
    \frac{1}{\rho}\nabla p = \frac{RT}{\mu}\nabla\,(\log(p))
\end{equation*}
\par For convenience's sake, we're going to switch to adimensional coordinates
\begin{equation*}
    \tilde{r} = r/r_{0} \quad \tilde{v} = v/\Omega_{0}r_{0} \quad \tilde{t} = \Omega_{0}t \implies \tilde{\nabla} = r_{0}\nabla
\end{equation*}
this way Euler's equation can be restated in terms of adimensional quantities only. Introducing once again the Virial temperature (\ref{eq:virial_temperature})
\begin{equation*}
    \frac{\partial \tilde{v}}{\partial \tilde{t}} + (\tilde{v}\cdot \tilde{\nabla})\cdot \tilde{v} = -\frac{T}{T_{vir}}\tilde{\nabla}\log(p)-\frac{1}{\tilde{r}^{2}}\hat{r}
\end{equation*}
\par From this last equation we're learning something important already: If $T \ll T_{vir}$ there's no feedback from the gas, and so the disk's geometry will be determined by gravity alone.
\par Assuming that there's little pressure pushing on the "vertical" direction, we may deduce that the vertical size will be "geometrically thin", and its dynamic will be dominated by gravity alone. In fact, at first order, the vertical component of the acceleration can be assumed to be
\begin{equation*}
    g_{z} = -\Omega_{0}^{2}z
\end{equation*}
which gives raise to a non-negligible velocity dispersion along the vertical direction. In short: Particles start oscillating.
\par Although we'd be contradicting ourselves, we're going to assume hydrostatic equilibrium to hold along the vertical direction. This allows us to find an analytic expression for the density profile along the vertical axis 
\begin{equation*}
    \rho(z) = \rho_{0}\exp\left(-\frac{z^{2}}{2H^{2}}\right)
\end{equation*}
with $H = c_{s}/\Omega_{0}$ the disk's scale height and $c_{s} = (TR/\mu)^{1/2}$ the sound speed. If we define the \emph{aspect ratio} $\delta$ of the disk as 
\begin{equation}
    \delta = \frac{H}{r} = \frac{c_{s}}{Hr} = \left(\frac{T}{T_{vir}}\right)^{1/2}= \frac{1}{M} < 1
    \label{eq:aspect_ratio}
\end{equation}
we conclude that the gas in the disk must be supersonic\footnote{For $\Omega_{0}r$ can be considered the tangential velocity to the (circular) orbit.}. Thus, the disk is pretty much a continuum of Keplerian orbits along which the gas has different velocities.
\par Now, if we move to the comoving frame and recall Fig.\ref{fig:stratified_fluid}, we may notice that we are in the exact conditions for shear instability to arise. But since angular momentum is \emph{fixed} for any given orbit, each displacement caused by shear instabilities will be simply ignored, for angular momentum conservation will eventually bring back the displaced element to its original position.
\par If that's the case, how can accretion be possible? Viscosity.
\par Although we have no clues (sorta\footnote{As for many other things in the Universe, it is probably turbulence's doing. We'll take a look at it in the next section.}\,) of how the viscosity gets high enough to allow the rings of gas to migrate towards the center and lose angular momentum, it's clear that if that is the case, then accretion may indeed be possible.
\section{The Shakura-Sunyaev model}
Coming soon.