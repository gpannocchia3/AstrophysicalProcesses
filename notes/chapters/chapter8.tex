\chapter{Black Holes Binaries and Coalescence}
\section{Introduction}
This last part of the notes will focus on (Astrophysical) Black Holes, going through the formation of BH binaries and all the processes that make possible the formation of such systems.
\par I'll closely follow my own notes, as well as those made available by Professor M. Crisostomi \cite{celoria2018lecturenotesblackhole}.
\vspace{0.5cm}
\par It is common to identify astrophysical black holes according to their mass
\begin{itemize}
    \item \emph{Stellar-mass Black Holes} (SBHs); they have masses in the range $M_{\odot}\lesssim M \lesssim 10^{2} M_{\odot}$. SBHs are what remains of stars with initial mass of the order $M\gtrsim 20M_{\odot}$.
    \item \emph{Intermediate-mass Black Holes} (IMBHs); they have masses in the range of $10^{2}M_{\odot}\lesssim M \lesssim 10^{6}M_{\odot}$. Since we have only scarce observative evidence about these guys, they're the ones we know the least about. We know that they can form at high redshift either from PopIII star remnants\footnote{"Population III stars are a hypothetical population of extremely massive, luminous and hot stars with virtually no "metals", except possibly for intermixing ejecta from other nearby, early population III supernovae. Such stars are likely to have existed in the very early universe (i.e., at high redshift) and may have started the production of chemical elements heavier than hydrogen, which are needed for the later formation of planets and life as we know it." Definition from \href{https://en.wikipedia.org/wiki/Stellar_population}{Wikipedia}.} or via direct collapse from a marginally stable protogalactic disk or quasistar, or at lower redshifts via dynamical processes in massive star clusters.
    \item \emph{(Super) Massive Black Holes} (MBHs or SMBHs); they have masses in the range of $10^{6}M_{\odot}\lesssim M \lesssim 10^{10}M_{\odot}$. They are observed at the center of almost all galaxies, AGNs and Quasars.
\end{itemize}
\begin{figure}[h!]
    \centering 
    \includegraphics[width=0.8\textwidth]{img/bh_massdistrib.png}
    \caption{Pictorial representation of the mass ranges of known black holes.\\ Credits: \cite{Colpi_2017}.}
\end{figure}
\par Most of the objects of astrophysical interest (stars, neutron stars, BHS etc.) are typically observed in binary systems. For example, half of the total amount of stars and, more interesting, 70\% of the stars with mass greater than ten solar masses, that are relevant to the formation of compact objects, live in binaries. On top of that, as shown in any GR course, two compact objects bound in a binary system have a non-zero quadrupole moment and are thus able to emit GWs.
\par After a short digression about redshift, more for the sake of completeness than anything, we'll start turning our attention to the physical features of binary systems. In particular, we'll consider the Keplerian motion of the binary, where two compact objects can be regarded as point masses and as long as the orbital velocities are small compared to the speed of light.
\par In short, unless pointed out otherwise, our discussion will be classical.
\subsubsection{A short detour on redshift}
In our description of the Universe we'd like to observationally determine a number of quantities to try to infer which of the models proposed by the FRW metric\footnote{As per any GR subject in this notes, I'll follow \cite{carroll}. You can find this discussion at §8.5.}
\begin{equation}
    \text{d}s^{2} = -\text{d}t^{2}+a(t)^{2}\left(\frac{\text{d}r^{2}}{1-kr^{2}}+r^{2}\text{d}\Omega_{(2)}^{2}\right)
    \label{eq:FRW}
\end{equation}
corresponds to our Universe. First and foremost, we could perform a slight manipulation of the FRW metric (\ref{eq:FRW}) and define a conformal time 
\begin{equation*}
    a(\eta)\text{d}\eta = \text{d}t
\end{equation*}
so that the metric is, guess what, conformal. The scale factor $a$ is therefore factorized and negligible if we consider null-like trajectories $\text{d}s^{2} = 0$. Note that in 
\begin{equation}
    \text{d}s^{2} = a(\eta)^{2}\left(-\text{d}\eta^{2}+\frac{\text{d}r^{2}}{1-kr^{2}}+r^{2}\text{d}\Omega_{(2)}^{2}\right)
    \label{eq:FRW_conf}
\end{equation}
the quantity within parentheses is independent of the conformal time. We could try with a hand-waving approach and obtain (out of thin air) the expression for the cosmological redshift, but we'll take a somewhat more rigorous approach.
\par Consider the four-velocity of comoving observers $U^{\mu} = (1,0,0,0)$, then the metric admits a Killing tensor of the form 
\begin{equation*}
    K_{\mu\nu} = a^{2}(g_{\mu\nu}+U_{\mu}U_{\nu})
\end{equation*}
that you can check that it satisfies $D_{(\sigma}K_{\mu\nu)} = 0$ where $D_{\sigma}$ is the covariant derivative. By definition of Killing vector (the generalization to a K. tensor is straightforward), the quantity 
\begin{equation}
    K^{2} = K_{\mu\nu}V^{\mu}V^{\nu} = a^{2}\left[V_{\mu}V^{\mu}+(U_{\mu}V^{\mu})^{2}\right]
    \label{eq:conserved_killing}
\end{equation}
is conserved along the geodesics. Here $V^{\mu} = \text{d}x^{\mu}/\text{d}\lambda$. For massive particles, since $V_{\mu}V^{\mu} = -1$ (note the convention on the signature of the metric), we also have
\begin{equation*}
    (V^{0})^{2} = 1+|\vec{V}|^{2} \quad |\vec{V}|^{2} = g_{ij}V^{i}V^{j}
\end{equation*}
Recalling the formula for the four-velocity of the observer, we have $U_{\mu}V^{\mu} = -V^{0}$, so (\ref{eq:conserved_killing}) implies
\begin{equation*}
    |\vec{V}| = \frac{K}{a}
\end{equation*}
Notice how this means that the particle "slows down" with respect to the comoving coordinates as the Universe expands. If we instead consider null geodesics (that is what we're actually interested in)
\begin{equation*}
    V_{\mu}V^{\mu} = 0 \implies U_{\mu}V^{\mu} = \frac{K}{a}
\end{equation*}
but for null-like particles, like photons, we can write the four-velocity as $V^{\mu} = (\omega, \vec{k})$. So the frequency of the photon as measured by a comoving observer is $\omega = -U_{\mu}V^{\mu}$ (remember that $c=1$). The frequency emitted will still be given by the same functional expression but calculated in the local inertial frame of the emitter, so $\omega_{em} = -K/a_{em}$. We then find out that 
\begin{equation*}
    \frac{\omega_{obs}}{\omega_{em}} = \frac{a_{em}}{a_{obs}}
\end{equation*}
To make things easier we then define a quantity called \emph{redshift} defined as 
\begin{equation*}
    z_{em} = \frac{\lambda_{obs}-\lambda_{em}}{\lambda_{em}}
\end{equation*}
so that, if the observation takes place today ($a_{obs} = a_{0} = 1$), we can just write 
\begin{equation*}
    a_{em} = \frac{1}{1+z_{em}}
\end{equation*}
and 
\begin{equation}
    \boxed{
    \frac{\nu_{obs}}{\nu_{em}} = \frac{1}{1+z_{em}}
    }
    \label{eq:cosmological_redshift}
\end{equation}
So the redshift tells us two things: The scale factor of the universe when the photon was emitted and, in a sense, the distance of the emitter when the photon was emitted. In fact, using Hubble's law (assuming we've somehow computed the Hubble constant at its present value) we can easily calculate the \emph{instantaneous physical distance}
\begin{equation*}
    v = cz = H_{0}d_{P}
\end{equation*}
Was all of this really necessary for what we'll be going through in the next sections? No, not really. But I thought it might have been cool having a clue on where this curious quantity called redshift comes from and, more importantly, knowing what it actually means and implies. That being said, we can hop on back on tracks.
\subsection{Keplerian Motion}
Let us consider two point-like objects of mass $M_{1}$, $M_{2}$ moving along elliptical orbits around their center of mass as shown in Fig.\ref{fig:keplerianmotion}.
\par As customary, we can define the total mass $M_{tot}$ and the reduced mass $\mu$ of the system as well as the relative coordinate $\vec{r} = \vec{r}_{1}-\vec{r}_{2}$, with $\vec{r}_{i}$ the distance of the mass $M_{i}$ from the center of mass.
\begin{figure}[h!]
    \centering 
    \includegraphics[width=0.8\textwidth]{img/keplerianmotion.png}
    \caption{Binary system religiously moving as predicted by Newtonian dynamics.\\
    Credits: \cite{celoria2018lecturenotesblackhole}.}
    \label{fig:keplerianmotion}
\end{figure}
\begin{equation*}
    \vec{r}_{1} = \frac{M_{2}}{M_{tot}} \vec{r} \quad \vec{r}_{2} = -\frac{M_{1}}{M_{tot}} \vec{r} 
\end{equation*}
The velocities relative to the center of mass can be obtained straightforwardly by deriving the expression above in respect to time.
\par The total energy of the system is 
\begin{equation*}
    E = \frac{M_{1}v_{1}^{2}}{2}+\frac{M_{2}v_{2}^{2}}{2} -\frac{GM_{1}M_{2}}{r} = \frac{\mu v^{2}}{2}-\frac{GM_{tot}\mu}{r} = -\frac{GM_{tot}\mu}{2a}
\end{equation*}
In the last step, we have used the conservation of energy along the orbit and wrote the energy in terms of the semi-major axis $a$. Hence, the binary system is equivalent to considering a single body with mass $\mu$ moving in an effective external gravitational potential. The orbital frequency is given by Kepler's third law 
\begin{equation*}
    \omega = \frac{2\pi}{T} = \sqrt{\frac{GM_{tot}}{a^{3}}}
\end{equation*}
which is, incidentally, the same result you obtain considering the (proper time) orbital period of a test-mass in the gravitational potential induced by a Schwartzschild metric\footnote{It's not difficult to show. You have to make use of the various conserved quantities the Killing vectors kindly gift you and then calculate $\text{d}\phi/\text{d}\tau$ with $\tau$ the proper time of the particle.}.
\par It is useful to calculate the orbital angular momentum $J_{orb} = \vec{r}\wedge\mu\vec{v}$, which has magnitude
\begin{equation*}
    J_{orb} = \mu\sqrt{GM_{tot}a(1-e^{2})}
\end{equation*}
where $e$ is the eccentricity of the orbit. Thus, for a circular orbit ($e = 0$) the magnitude of the angular momentum is just 
\begin{equation*}
    J_{orb, C} = \mu\sqrt{GM_{tot}r_{C}}
\end{equation*}
hence the tangential velocity is
\begin{equation*}
    v = \sqrt{\frac{GM}{r_{C}}}
\end{equation*}
\subsection{GW radiation from a binary system}
As you should probably know if you've attended a GR class, a system of sufficiently\footnote{Please note that there's not a threshold, proper, on the mass or density of the objects to start emitting GW, but rather their intensity would be so faint that we couldn't possibly dream about observing them.} compact objects bound in a binary system can start emitting GWs. A full description of how GWs are produced in the weak field limit from the linearized Einstein's equation is beyond the scope of these notes. The interested reader can refer to \cite{carroll}, §7.
\par For Newtonian sources localized in a compact region of space, the gravitational power radiated is governed by the \emph{quadrupole momenta} of the binary and given by 
\begin{equation}
    \frac{\text{d}E_{rad}}{\text{d}t} = \frac{G}{5}\langle\dddot{I_{ij}}\dddot{I_{ji}}\rangle
    \label{eq:quadrupole_rad}
\end{equation}
where the brackets stand for average over the solid angle and the tensor $I_{ij}$ is the mass quadrupole moment given by the following integral ($c=1$)
\begin{equation*}
    I_{ij}(t) = \int_{\text{source}} \text{d}^{3}r\,\left(r_{i}r_{j}-\frac{1}{3}\delta_{ij}r^{2}\right)T_{00}
\end{equation*}
\par For binary systems, it is possible to prove that the averaged power emitted can be expressed as 
\begin{equation*}
     \frac{\text{d}E_{rad}}{\text{d}t} = \frac{32}{5}\frac{G^{4}}{c^{5}}\frac{\mu^{2}M^{3}}{a^{5}}F(e)\
\end{equation*}
where the factor 
\begin{equation*}
    F(e) = (1-e^{2})^{-7/2}\left(1+\frac{73}{24}e^{2}+\frac{37}{96}e^{4}\right)
\end{equation*}
depends only on the eccentricity and shows that highly eccentric binaries are much more efficient at radiating away energy in form of GWs.
\par We can also compute the averaged angular momentum flux 
\begin{equation*}
    \frac{\text{d}J_{rad}}{\text{d}t} = \frac{32}{5}\frac{G^{7/2}}{c^{5}}\frac{M_{1}^{2}M_{2}^{2}(M_{1}+M_{2})^{1/2}}{a^{7/2}}(1-e^{2})^{-2}\left(1+\frac{7}{8}e^{2}\right)
\end{equation*}
From these equations we see that GWs extract both energy and angular momentum out of the binary; as a consequence, the two masses are drawn closer, spiraling around each other until they eventually merge.
\par From the previous equations is possible to show that 
\begin{align}
    \frac{\text{d}a}{\text{d}t} &= -\frac{64}{5}\frac{G^{3}}{c^{5}}\frac{M_{1}M_{2}(M_{1}+M_{2})}{a^{3}}F(e)\\
    \frac{\text{d}e}{\text{d}t} &= -\frac{304}{15}\frac{G^{3}}{c^{5}}\frac{M_{1}M_{2}(M_{1}+M_{2})}{a^{4}}e(1-e^{2})^{-5/2}\left(1+\frac{121}{304}e^{2}\right)
    \label{eq:orb_variation}
\end{align}
from which we see that GWs drive the binary towards circularization along the inspiral.
\par We can integrate in time the expression for the separation and, neglecting that the eccentricity varies in time, find an estimate for the \emph{coalescence time} of the binary 
\begin{equation}
    t_{\text{coal}} = \frac{5}{256}\frac{c^{5}}{G^{3}}\frac{a_{0}^{4}}{M_{1}M_{2}(M_{1}+M_{2})}\frac{1}{F(e)}
    \label{eq:coalescence_time}
\end{equation}
We can then rewrite the initial separation $a_{0}$ as 
\begin{equation*}
    a_{0} = 1.6R_{\odot}\left(\frac{M_{1}}{M_{\odot}}\right)^{3/4}\left[q(1+q)F(e)\left(\frac{t_{\text{coal}}}{1\,\text{Gyr}}\right)\right]^{1/4}
\end{equation*}
where $q = M_{2}/M_{1}$. From the expression above it's immediate to see that the initial separation required for two objects bound in a binary to merge within a Hubble time\footnote{The Hubble time, if you recall, is $t_{H} = H_{0}^{-1}\approx 14.4\,\text{Gyr}$ and is a decent approximation of the age of the Universe.} is $a_{0}\approx 10^{11}\,\text{cm}\approx 0.01\,\text{AU}$, which is quite the narrow range, actually.
\par This is also quite problematic if you think about it. For once, during their evolution, stars may undergo a giant phase characterized by $R_{G} \approx 10^{14}\,\text{cm}$, so stars with initial separation $a_{0}<R_{G}$ would simply engulf each other during their giant phase and merge in a single star \emph{before} forming our nice binary of compact objects.
\par The scope of the next sections is trying to find a way to have our compact objects come sufficiently close so that they can get bound in a binary system and then, eventually, merge.
\section{Common Evolution Channel (SBHs)}
As you may have guessed from the title of this section, we're now turning our attention to stellar-mass BHs; in the following, we'll try to elaborate and write down a sensible mechanism to bring the two objects close enough to form our cute binary. Two formation channels consistent with
the first GWs observations of SBH binaries mergers have been proposed: the \emph{common evolution of field binaries} and the \emph{dynamical capture in dense
environments}. Let's focus on the first one.
\par The common evolution is the astrophysical scenario in which the two stars, eventually producing the SBH binary, form as a stellar binary system and evolve together through the different phases of stellar evolution. To achieve this, we have to look into four key ingredients, which will be the subject of the four next subsections.
\subsection{Gravitational potential}
As we've already done somewhere above, let us consider two point-like objects of mass $M_{1}$ and $M_{2}$ separated by $a$ and moving, for simplicity, in circular orbits about their common center of mass. In the corotating frame, the energy potential of the system is just 
\begin{align*}
    U &= -\frac{Gm(M_{1}+M_{2})}{r_{CM}}-\frac{1}{2}m\omega^{2}r_{CM}^{2}\\
    &= -Gm\left(\frac{M_{1}}{s_{1}}+\frac{M_{2}}{s_{2}}\right)-\frac{1}{2}m\omega^{2}r_{CM}^{2}
\end{align*}
where $s_{i}$ is the distance of a test-mass $m$ from mass $M_{i}$ and $r_{CM}$ is the distance of said test-mass from the center of mass.
\par As we've already seen in the last chapter, the equilibrium points are the Lagrange points of the potential, which satisfy 
\begin{equation*}
    \vec{F} = -m\nabla\,\phi = 0
\end{equation*}
There exists a critical equipotential surface, forming a two-lobed figure-of-eight, with one of the two object at the centre of each lobe and intersecting itself at the $L_{1}$ Lagrangian point, known as Roche lobes (see Fig.\ref{fig:roche}). An approximate formula for the radius of the Roche lobe around $M_{1}$ was derived by Eggleton 
\begin{equation*}
    \frac{R_{1}}{a} = \frac{0.49q^{2/3}}{0.6q^{2/3}+\ln(1+q^{1/3})}
\end{equation*}
up to a 1\% accuracy.
\subsection{Mass transfer}
When one of the two objects fills its Roche lobe, matter may overflow the lobe and infall onto the other object, without any need of energy exchange.
\par Suppose that $M_{2}$ loses material at a rate $\dot{M}_{2}<0$ and let $\beta \in [0,1]$ be the fraction of the ejected matter leaving the system, so that 
\begin{equation*}
    \dot{M}_{1} = -(1-\beta)\dot{M}_{2} \geq 0
\end{equation*}
Clearly, if $\beta = 0$, then all the mass lost by $M_{2}$ is captured by $M_{1}$ and the mass transfer is fully conservative. Keeping it as a free parameter, we are instead considering the more
general case in which a fraction of the mass can be lost and escape the system, as may be the case, for example, in the presence of stellar winds.
\par The angular momentum was $J = \mu\sqrt{GMa}$; if we now differentiate it with respect to time and use $\dot{M}_{1}+\dot{M}_{2} = \dot{M} = \beta\dot{M}_{2}$, we obtain 
\begin{equation}
    \frac{\dot{a}}{a} = -2\left(1-\frac{M_{2}}{M_{1}}\right)\frac{\dot{M}_{2}}{\dot{M}_{1}}
\end{equation}
In short, if $M_{1}>M_{2}$, the orbit expands ($\dot{a}>0$), otherwise it shrinks. From Kepler's third law we also know that 
\begin{equation*}
    \frac{\dot{\omega}}{\omega} = -\frac{3}{2}\frac{\dot{a}}{a}
\end{equation*}
and thus the angular frequency unsurprisingly increases as the orbit shrinks.
\subsection{Supernova kicks}
At the end of all the subsequent stages of nuclear burning\footnote{Note that nothing is \emph{actually} burning in the common sense of "chemical burning". It's just an astrophysical slang to refer to nuclear fusion.}, there's a chance that the more massive stars can undergo a supernova (SN) explosion\footnote{As shown in Chapter \ref{ch:v}, §6.3.2, SN explosions are described fairly well with the Sedov-Taylor blastwave model.}, after which much of the stellar material is expelled and leaving behind only a shell of its former self, typically in the form of a neutron star or a stellar-mass black hole. The mass loss is practically instantaneous as the typical timescale for the explosion is much shorter than the orbital period.
\par In general, the collapse is not perfectly
symmetric nor isotropic. As a result, the SN imprints a kick to the object characterised by a recoil velocity $v_{\text{kick}}$. Since mass ejection decreases the total mass of the binary, also the gravitational potential changes and, if enough mass is ejected, the SN explosion can unbind the binary (which would suck for our purposes' sake).
\par Moreover, $v_{\text{kick}}$ is generally (much) greater than the orbital velocity, so the kick may as well destroy most of the binaries.
\par To describe the effect of a SN explosion, we start by considering the setup we've used before, the notation is identical.
\par Prior to the SN-explosion, the relative velocity is 
\begin{equation*}
    v_{i} = \sqrt{\frac{G(M_{1}+M_{2})}{a_{i}}}
\end{equation*}
After the explosion of, say, the giant star $M_{1}$, what remains of that star has mass $M_{c}<M_{1}$, and $\Delta M = M_{1}-M_{c}$ is ejected. We can assume that, after an instantaneous explosion, the position of the once-$M_{1}$ has not changed, but the reduced mass has
\begin{equation*}
    \mu_{f} = \frac{M_{c}M_{2}}{M_{c}+M_{2}}
\end{equation*}
and the final (relative) velocity is $\vec{v}_{f} = \vec{v}_{i}+\vec{v}_{\text{kick}}$. The final energy of the system will then just be
\begin{equation*}
    E_{f} = \frac{1}{2}\mu_{f}v_{f}^{2}-\frac{GM_{c}M_{2}}{a_{i}}
\end{equation*}
The system will remain bound only if the final velocity is smaller than the escape velocity
\begin{equation*}
    v_{f}\leq v_{e} = \sqrt{\frac{2G(M_{c}+M_{2})}{a_{i}}}
\end{equation*}
If the SN-explosion is perfectly spherically symmetric, isotropy leads to having no kick at all, and the condition for "boundness" implies 
\begin{equation*}
    \Delta M \leq \frac{M_{1}+M_{2}}{2}
\end{equation*}
as it's required from imposing $E_{f}<0$ under the assumption of $v_{i}=v_{f}$.
Even in absence of kicks, a binary can be disrupted due to mass loss only, if the SN explosion ejects more than half of the initial mass of the binary
system.
\par The case of asymmetric SNs is more complicated and usually characterized by kicks of order $10^{2}\,\text{km}/\text{s}$. More precisely, the natal
kick distribution is typically modelled by a Maxwellian probability distribution with velocity dispersion $\sigma_{v} = 190\,\text{km}/\text{s}$, even though, a bimodal distribution may be somewhat more appropriate, especially when describing neutron stars' kicks.
\par For SBHs the situation is less clear and the problem of binary disruption due to SN kicks might be less severe. In this case, there is no hard surface to bounce onto and, as a consequence, the kicks might be smaller, typically of order of $50\,\text{km}/\text{s}$, which is, however, of the order of the the escape velocity for most globular clusters in the Milky Way.
\section{Common Envelope}
Giant stars are composed of a core and an envelope, and to a good approximation we may think of decomposing the mass of the star as $M_{\text{gs}} = M_{\text{core}} + M_{\text{env}}$. In the core, Hydrogen has been fully converted into Helium and the nuclear reactions have (momentarily) stopped, causing a core contraction under the action of gravity. The envelope, on the other hand, is mainly made of Hydrogen. It is then safe to assume, and treat, the two objects as well-separated objects.
\par When the giant star overfills its Roche lobe, mass transfer is allowed and, depending on the mass of the two stars, the orbit may shrink causing even more material to overflow the Roche lobe. This eventually leads to the runaway process of dynamically unstable mass transfer.
\par It is therefore possible that the mass transfer rate from the donor is so high that the SBH cannot accommodate all the accreting matter. In this situation, the envelope continues to expand, eventually engulfing the companion SBH and leading to the formation of a \emph{common envelope}. The common envelope can extract energy from the orbit of the binary system, formed by the SBH and the core of the massive star, via dynamical friction, eventually unbinding itself from the system.
\par We can calculate the initial binding energy of the stellar envelope 
\begin{equation*}
    E_{\text{env},\,i} = -\frac{GM_{\text{gs}}M_{\text{env}}}{\lambda R_{L}}
\end{equation*}
where $R_{L}$ is the typical Roche lobe's radius, while $\lambda$ is the concentration parameter that depends on the density profile of the envelope. Usually, the density of the envelope may be well-described as a power law $\rho(r) \propto r^{-\gamma},\,\gamma > 0$. If we assume that, at the end of the common envelope stage, the envelope unbinds, formally reaching infinity with zero velocity, we can write
\begin{equation*}
    \Delta E_{\text{env}} = -E_{\text{env},\,i} = \frac{GM_{\text{gs}}M_{\text{env}}}{\lambda R_{L}}
\end{equation*}
In short, the more concentrated the envelope is the more binding energy is possible to extract from it. As a consequence, the separation between the SBH and the core of the giant star decreases. We can compute the
variation of the binary orbital energy
\begin{equation}
    \Delta E_{\text{orb}} = \alpha_{ce}\left[-\frac{GM_{\text{core}}M_{\text{BH}}}{2a_{f}}-\left(-\frac{G(M_{\text{core}}+M_{\text{env}})M_{\text{BH}}}{2a_{i}}\right)\right]
    \label{eq:boe_var}
\end{equation}
where $\alpha_{ce}$ is the common envelope parameter describing the efficiency of the expenditure of orbital energy upon expulsion of the envelope.
\par Since conservation of energy implies 
\begin{equation*}
    \Delta E_{\text{orb}}+\Delta E_{\text{env}} = 0
\end{equation*}
if we use the core-envelope decomposition, we can calculate the ratio between the initial and final semi-major axes of the orbit 
\begin{equation}
    \frac{a_{f}}{a_{i}} = \frac{M_{\text{core}}}{M_{\text{gs}}}\left(1+\frac{2}{\lambda\alpha_{ce}}\frac{a_{i}}{R_{L}}\frac{M_{\text{env}}}{M_{\text{BH}}}\right)^{-1} = \frac{M_{\text{core}}}{M_{\text{gs}}}\left(\frac{M_{\text{BH}}}{M_{\text{BH}} + \frac{2M_{\text{env}}}{\lambda\alpha_{ce}}\frac{a_{i}}{R_{L}}}\right)
    \label{eq:sma_ratio}
\end{equation}
Despite the large uncertainties, it is possible to estimate the product $\lambda\alpha_{ce}$ by modelling specific systems or well-defined samples of objects corrected for observational selection effects. In general, the envelope is quite concentrated and typical values are of the order $\lambda\alpha_{ce}\ll 1$ and $M_{\text{core}}/M_{\text{gs}} \approx 0.2-0.3$. Thus, the order of magnitude approximation of (\ref{eq:sma_ratio}) is 
\begin{equation*}
    \frac{a_{f}}{a_{i}} \sim 10^{-3}-10^{-2}
\end{equation*}
\par This process takes about $10^{2}-10^{3}$ years and upon completion the system has shrunk to $a_{f}\approx R_{\odot}$, close enough to merge in a Hubble time due to GW emission.
\par What we've described so far works particularly well to predict what happens when all other gravitational sources are absent.
\par Let's assume that the average stellar mass density in the Universe is of order $\rho_{*} \approx 3\cdot10^{8}\,M_{\odot}/\text{Mpc}^{3}$; from Salpeter's initial mass function (IMF) we know that $n\sim M^{-2.35}$. This implies that roughly 0.3\% of the total stars in the Universe are over $30M_{\odot}$, thus leaving a SBH as relic at the end of their life. Since 70\% of the massive stars are observed in binaries, and we need two stars to form a binary, we can give an estimate and claim that there are about $3\cdot 10^{5}$ massive binary stars per cubic megaparsec.
\par We can make the (extreme) assumption that those binaries are produced in a continuous, steady-state star formation process over about 10 Gyr, thus resulting in a formation rate of $3\cdot 10^{-5}\,\text{yr}^{-1}\,\text{Mpc}^{-3}$; assuming that all those massive binaries give rise to SBHBs that merge in a short timescale, then we get a SBHB merger rate\,\footnote{Note that this doesn't take into account the binary systems that get disrupted by SN kicks, so the number of mergers will be rather lower. On top of that, we also assumed a constant star formation rate across cosmic time, although it's known that it peaks at about $z\approx 1.5$.} of $3\cdot 10^{4}\,\text{yr}^{-1}\,\text{Gpc}^{-3}$.
\section{BH Binaries in Globular Clusters}
Another channel to form SBH binaries is via dynamical processes; this is closely related to the fact that most stars are observed to form in clusters and associations. However, the vast majority of the stars we observe today in the MW\footnote{Short for Milky Way.} are field stars and therefore do not belong to stellar associations\footnote{A "stellar association" is a group of more than $10^{3}$ stars, like globular clusters, young massive star clusters and open clusters.}.
\par In order to understand the dynamical formation scenario, it is therefore important to start from the physics of star cluster formation and
evolution.
\par Let's start with the basics.
\subsection{Jeans' mass}
Consider a cloud of gas. The virial theorem states that in order to achieve and maintain hydrostatic equilibrium the following relation must hold 
\begin{equation}
    -2\langle K\rangle = \langle U\rangle
    \label{eq:virial_th}
\end{equation}
where brackets now denote time-averaged quantities. From here it's clear that the condition for gravitational collapse is simply 
\begin{equation*}
    2\langle K\rangle < \langle |U|\rangle
\end{equation*}
Under the assumption of spherical symmetry, we can write the infinitesimal gravitational potential energy over a thin shell of mass $\text{d}m$
\begin{equation*}
    \text{d}U = -\frac{GM(r)\text{d}m}{r} = -GM(r)4\pi\rho r\text{d}r
\end{equation*}
If density is constant, then $M(r) = 4\pi\rho r^{3}/3$ and the mass of a globular molecular cloud of given radius $R$ is $M = 4\pi\rho R^{3}/3$. Integrating the potential energy yields 
\begin{equation*}
    U = -\frac{3}{5}\frac{GM^{2}}{R}
\end{equation*}
The total internal energy of the cloud is $K = 3NkT/2$. We can define the mean molecular weight as 
\begin{equation*}
    \mu = \frac{\langle m \rangle}{m_{H}}
\end{equation*}
so that $N = M/\mu m_{H}$. The condition for gravitational collapse can therefore be rewritten as 
\begin{equation*}
    \frac{3MkT}{\mu m_{H}} < \frac{3}{5}\frac{GM^{2}}{R}
\end{equation*}
Solving for $M$ and plugging back the expression for the radius of a spherical and constant distribution of matter, we obtain 
\begin{align}
    M > M_{J} &= \left[\frac{375}{4}\left(\frac{k}{Gm_{H}}\right)^{3}\frac{T^{3}}{\mu^{3}\rho}\right]^{1/2} \\
    &\approx 3\cdot 10^{3}M_{\odot}\left[\frac{1}{\mu^{4}}\left(\frac{T}{100\,\text{K}}\right)^{3}\left(\frac{10^{3}\,\text{cm}^{-3}}{n}\right)\right]^{1/2}
    \label{eq:jeans_mass}
\end{align}
where we've used $\rho = n\langle m \rangle = n\mu m_{H}$. If the mass of the cloud exceeds the Jeans' mass, the cloud will be unstable against gravitational collapse.
\subsection{Free-fall timescale}
Let's consider a spherical cloud with constant density. In presence of gravity alone, we can write 
\begin{equation*}
    (\dot{r})\ddot{r} = -\frac{GM}{r^{2}}(\dot{r})
\end{equation*}
which can be immediately integrated to yield 
\begin{align*}
    \frac{1}{2}\dot{r}^{2} &= \frac{GM}{r}+\text{const.}\\
    &= \frac{GM}{r}-\frac{GM}{R}
\end{align*}
where we have set the initial conditions for $t=0$: $r=R$ and $\dot{r} = 0$.
\par The free-fall timescale can be thus obtained by committing a mathematical atrocity, but I (and, for extension, you too) don't care\footnote{It is actually less atrocious than I'm making it sound like.}
\begin{align*}
    \tau_{ff} &= \int_{0}^{t_{ff}} \text{d}t = \int_{R}^{0}\frac{1}{\dot{r}}\,\text{d}r \\
    &= \int_{0}^{R} \frac{\text{d}r}{\sqrt{2GM\left(\frac{1}{r}-\frac{1}{R}\right)}} = \frac{\pi}{2\sqrt{2}}\sqrt{\frac{R^{3}}{GM}}
\end{align*}
thus, if we use $R^{3}/GM = 3/4\pi G\rho$
\begin{equation}
    \tau_{ff} = \sqrt{\frac{3\pi}{32G\rho}} \approx 1.63\,\text{Myr}\cdot\left(\frac{10^{3}\,\text{cm}^{-3}}{\mu n}\right)^{1/2}
    \label{eq:ff_timescale}
\end{equation}
It shouldn't surprise you that under this approximation, all points get to the center at the same time, and density \emph{at all points} increases at the same rate. This is clearly unphysical, but serves well as a first approach to our system.
\par Similarly, we can calculate the explosion time, or sound time, as 
\begin{equation}
    \tau_{\text{sound}} = \frac{R}{c_{s}} \approx 0.5\,\text{Myr}\,\left(\frac{R}{0.1\,\text{pc}}\right)\left(\frac{0.2\,\text{km}\,\text{s}^{-1}}{c_{s}}\right)
    \label{eq:sound_time}
\end{equation}
If $\tau_{ff}>\tau_{\text{sound}}$ the system returns to a stable equilibrium as the pressure forces can overcome gravity. Conversely, for $\tau_{ff}<\tau_{\text{sound}}$ gravitational collapse takes place.
\par We can define the Jeans' length as the distance over which (in spherical symmetry) is containted a Jeans' mass 
\begin{equation*}
    R = \lambda_{J} = c_{s}\sqrt{\frac{3\pi}{32G\rho}}
\end{equation*}
More precisely, if we were to involve relativistic corrections, the Jeans' length is 
\begin{equation*}
    \lambda_{J} =  c_{s}\sqrt{\frac{\pi}{G\rho}}
\end{equation*}
\subsection{Cloud fragmentation}
Jeans' instability plays a very important role in star formation, because it is responsible for the fragmentation of the molecular cloud. We've seen that the Jeans' mass depends on the temperature and the density 
\begin{equation*}
    M_{J}\propto \left(\frac{T^{3}}{\rho}\right)^{1/2}
\end{equation*}
Suppose that the cloud is big and massive enough that potential energy overcomes internal energy and the cloud starts to collapse. During the collapse, $R$
decreases and $T$ increases. The process can be either \emph{isothermal} or \emph{adiabatic}, depending on whether the gas can efficiently cool or not, which is ultimately
related to the metallicity content of the gas.
\par Whether a cloud can cool efficiently or not depends on its metallicity. This is because at typical temperatures ($T \approx 100\,\text{K}$), collisions in the gas cannot excite the atomic levels of Hydrogen and Helium ($T > 10^{3}\,\text{K}$ is needed). Heavier elements, however, have much smaller energy gaps between atomic states. So, also at low temperatures, collisions in the gas excite atomic/molecular levels that return to the ground state by emitting photons that can effectively escape, thus cooling the cloud. Such cooling is possible if the metallicity $Z > Z_{\text{crit}}$ with $10^{-4}\lesssim Z_{\text{crit}}/Z_{\odot}\lesssim 10^{-3}$.
\subsubsection*{Adiabatic collapse}
If the metallicity is below the critical threshold, the cloud cannot cool efficiently and the collapse is approximately adiabatic. So 
\begin{equation}
    pV^{\gamma} = \text{const.} \quad \text{or}\quad  p = K\rho^{\gamma}
    \label{eq:adiabatic_relation}
\end{equation}
with $\gamma$ the adiabatic index of the gas and $K$ a constant. Using the ideal gas EoS we find a relation between temperature and density
\begin{equation*}
    T \propto \rho^{\gamma-1}
\end{equation*}
so the Jeans' mass is 
\begin{equation*}
    M_{J} \propto \left(\frac{T^{3}}{\rho}\right)^{1/2}= \rho^{3\gamma/2-2}
\end{equation*}
For atomic Hydrogen $\gamma = 5/3$ and $M_{J}\propto \rho^{1/2}$, so that the Jeans' mass increases as the density increases. As a consequence, the cloud cannot fragment and the collapse is approximately \emph{monolithic}\footnote{The Jeans mass sets a lower bound to the mass of an object to be stable against gravitational collapse. Hence, if $M_{J}$ increases as the cloud collapses (for $\rho$ is always increasing during a collapse), only objects with mass $M>M_{J}$ may be subject to gravitational collapse. In short, only the cloud as a whole can collapse.}; in other words, during an adiabatic collapse, the cloud contracts without fragmenting,  thus potentially forming a single massive protostar of the order of $10^{3}M_{\odot}$. The reason for this upper limit comes from the fact that, when the cloud collapses adiabatically, the temperature eventually increases up to $T\approx 10^{2}\,\text{K}$ ($M_{J}\approx 10^{3}M_{\odot}$); at that point, molecular hydrogen transitions start to cool the cloud, and the Jeans' mass does not increase anymore.
\par Adiabatic collapse is held responsible as one of the main formation channels of IMBHs. Remarkably, early on in structure formation, at redshift $15 < z < 20$, the gas metallicity was very low, so the first protostellar clouds likely collapsed with little fragmentation, leaving behind a first generation of stars (PopIII stars) that were very massive and could potentially evolve into IMBHs characterized by mass of the order of $10^{2}-10^{3}M_{\odot}$.
\subsubsection*{Isothermal collapse}
Conversely to what we've seen in the last section, for $Z$ over the critical threshold the collapse is essentially isothermal: Temperature remains approximately constant and the mass limit for instability decreases when the density of the cloud increases as $M_{J}\propto \rho^{-1/2}$. Any initial density perturbations will then cause individual regions within the cloud to cross the instability threshold independently and collapse locally, forming a large number of smaller objects: Isothermal
collapse naturally leads to \emph{fragmentation}.
\par It is obvious, however, that in order for protostars to \emph{actually} form, the process cannot remain isothermal indefinitely, for temperatures of order $T\sim 10^{7}\,\text{K}$ are needed to turn on nuclear reactions.
\par This happens when the clump density becomes high enough that gas becomes opaque to infrared photons. At this point, the cloud cannot cool efficiently, the isothermal approximation breaks down and further evolution is approximately adiabatic. As a consequence, temperature increases
and so does $M_{J}$, thus leading to a minimum fragment size into which the cloud can break up. This is known as the \emph{opacity limit}.
\subsubsection*{Opacity limit}
The energy released during the collapse of the protostellar cloud obeys the virial theorem (\ref{eq:virial_th}), so 
\begin{equation*}
    \langle E \rangle = \langle U \rangle + \langle K \rangle = \frac{\langle U \rangle}{2}
\end{equation*}
and only half the change in gravitational potential energy can be radiated away. Since (assuming spherical symmetry)
\begin{equation*}
    U = -\frac{3}{5}\frac{GM^{2}}{R}
\end{equation*}
the energy released is $\Delta E_{g}=3GM^{2}/10R$, from which we can calculate the emitted (average) luminosity over the free-fall time 
\begin{equation}
    L_{ff} = \frac{\Delta E_{g}}{\tau_{ff}} = \frac{3}{10}\frac{GM^{2}}{R}\left(\frac{3\pi}{32G\rho}\right)^{-1/2} = \frac{3\sqrt{2}}{5\pi}G^{3/2}\left(\frac{M}{R}\right)^{5/2}
    \label{eq_luminosity_opacitylimit}
\end{equation}
This is the power that, in the opacity limit, is absorbed by the gas. For the sake of simplicity, we're going to assume that the gas emits radiation as a grey-body, thus the cloud can radiate away energy at a rate given by 
\begin{equation*}
    L_{\text{rad}} = 4\pi R^{2}e_{f}\sigma T^{4}
\end{equation*}
where we have introduced the efficiency factor $e_{f}$ ($0 < e_{f} < 1$), because the collapsing cloud is not in thermodynamic equilibrium. The gas will start to
heat up (becoming adiabatic) as soon as the cooling becomes slower than the heating rate
\begin{equation*}
    L_{rad} \lesssim L_{ff}
\end{equation*}
The isothermal approximation then breaks down where the two luminosities are equal 
\begin{equation}
    M^{5}\lesssim M^{5}_{\text{crit}} = \frac{200}{9}\frac{\pi^{4}\sigma^{2}}{G^{3}}e_{f}^{2}R^{9}T^{8}
    \label{eq:isothermal_breakdown}
\end{equation}
This gives us a further constraint on the mass of the cluster: $M_{J}<M<M_{\text{crit}}$. The opacity limit is reached when $M_{J} = M_{\text{crit}}$. For typical values of $\mu = 1$, $e_{f} = 0.1$, $T = 10^{3}\,\text{K}$ we have a mass for the single fragment of order $M_{\text{frag}} \approx 0.2M_{\odot}$. Hence, fragmentation ceases when individual fragments are approximately solar-mass objects.
\section{Evolution of Star Clusters}
Consider an idealized cluster of size $R$ consisting of $N$ (for simplicity) identical stars with mass $m$ uniformely distributed.
\input{chapters/impactparamter.tex}
\subsection{Two-body relaxation timescale}
Given this setup, depicted in Fig.\ref{fig:star_impact}, we're now going to show that stars in clusters can reach equilibrium through mutual interactions in a process named \emph{two-body relaxation}, which is essentially analogous to thermalization. Therefore, a very important timescale in collisional dynamics in clusters is the time for a star to completely lose memory of its initial velocity by means of gravitational encounters.
\par The gravitational force is 
\begin{equation*}
    \vec{F} = -\frac{Gm^{2}}{r^{3}}\vec{r} = \vec{F}_{\parallel}+\vec{F}_{\perp}
\end{equation*}
where the orthogonal component is
\begin{equation*}
    F_{\perp} = -\frac{Gm^{2}}{r^{2}}\frac{b}{r} = -G\frac{m^{2}}{b^{2}}\left[1+\left(\frac{vt}{b}\right)^{2}\right]^{-3/2} = -m\dot{v}_{\perp}
\end{equation*}
since $r^{2} = b^{2}+(vt)^{2}$. The change in velocity integrated over one entire encounter is 
\begin{equation*}
    \delta v_{\perp} = \int_{-\infty}^{\infty}\dot{v}_{\perp}\,\text{d}t = \frac{Gm}{b^{2}}\int_{-\infty}^{\infty}\left[1+\left(\frac{vt}{b}\right)^{2}\right]^{-3/2}\,\text{d}t
\end{equation*}
Note that for symmetry reasons it must be $\delta v_{\parallel} = 0$. The integral above is trivial and evaluates to\footnote{Before you slander me: Set $vt/b = \sinh(y)$, then unpack the hyperbolic cosine that pops out and set $z = e^{2y}$. The integral is indeed trivial and evaluates to $2b/v$.}
\begin{equation*}
     \delta v_{\perp} = \frac{2Gm}{bv}
\end{equation*}
Now taking into account all stars in the system, the surface density of stars in an idealized cluster is $N/\pi R^{2}$, and the number of interactions per unit element is 
\begin{equation*}
    \delta n = \frac{N}{\pi R^{2}}\text{d}(\pi b^{2}) = \frac{2bN}{R^{2}}\text{d}b
\end{equation*}
Defining $\delta v^{2}_{\text{tot}} = \int \delta v^{2}_{\perp}\,\delta n$, where the integral is performed over all possible impact parameters, we get
\begin{equation*}
    \delta v^{2}_{\text{tot}} = 8N\left(\frac{Gm}{Rv}\right)^{2}\log\left(\frac{b_{\text{max}}}{b_{\text{min}}}\right)
\end{equation*}
The integration limit $b_{\text{max}}$ is of the order of the size of the system, while $b_{\text{min}}$ corresponds to the smallest $b$ to avoid stellar collisons, which can be shown to be equal to 
\begin{equation*}
    b_{\text{min}} = \frac{2Gm}{v^{2}}
\end{equation*}
thus leading to 
\begin{equation*}
     \delta v^{2}_{\text{tot}} = 8N\left(\frac{Gm}{Rv}\right)^{2}\log\left(\frac{Rv^{2}}{2Gm}\right)
\end{equation*}
The typical speed of a star in a virialized system\footnote{We're referring to stars that obey the virial theorem (\ref{eq:virial_th}) and the equation that follows is merely a restating of the more general theorem.} is given by
\begin{equation*}
    Nmv^{2} = \frac{G(Nm)^{2}}{R}
\end{equation*}
Replacing for $v$ in the equation for $\delta v_{\text{tot}}$ we get 
\begin{equation*}
    \delta v^{2}_{\text{tot}} = v^{2}\frac{8}{N}\log\left(\frac{N}{2}\right)
\end{equation*}
The number of crossings for which $\delta v^{2}_{\text{tot}}/v^{2} \approx 1$ (that is the number of crossings after which the star has changed its initial velocity completely and lost memory of its initial conditions) is given by
\begin{equation*}
    n_{\text{cross}}\approx\frac{N}{8}\frac{1}{\log(N/2)}
\end{equation*}
The time needed to cross the system is 
\begin{equation*}
    \tau_{\text{cross}} = \frac{R}{v} = \sqrt{\frac{R^{3}}{GNm}} \propto \frac{1}{\sqrt{G\rho}}\propto \tau_{ff}
\end{equation*}
while the time necessary for stars in a system to completely lose the memory of their initial velocity, called the relaxation time, is
\begin{equation}
    \tau_{rlx} = n_{\text{cross}}\tau_{\text{cross}} = \frac{N}{8}\frac{1}{\log(N/2)}\frac{R}{v}\approx 10^{7}-10^{10}\,\text{yr}
    \label{eq:relaxation_time}
\end{equation}
Collisional systems have $\tau_{rlx}\ll t_{\text{Hubble}}$ and are therefore "relaxed" by two body interactions to a state that does not retain memory of their initial conditions.
\subsection{Infant mortality}
Most stars form in clusters, but nowadays we're observing only a minority of stars in clusters and associations. This happens because most clusters actually dissolve as they go through certain processes, as, for example, infant mortality.
\par Clusters are essentially bound systems of stars and gas, as so, they're not immune to aging\footnote{So to say.}. As the more massive stars go through their life cycle, it often happens that stellar winds or, more dramatically, SN explosions disrupt the cluster and blow away its internal gas, which won't ever be turned into new stars. This leads to a decreasing of the total gravitational potential holding the stars together, and so stars can escape in an irreversible runaway process that can completely dissolve the cluster.
\par For a cluster with mass $M = M_{g}+M_{*}$, if the mass of the gas getting lost is $M_{g}>M_{*}$, then the cluster dissolves. From the virial theorem, the velocity dispersion of the stars before gas removal is estimated as 
\begin{equation*}
    \sigma_{0}^{2} = \frac{G(M_{g}+M_{*})}{R_{0}}
\end{equation*}
where $R_{0}$ is the initial size of the cluster. Assuming instantaneous gas removal
\begin{equation*}
    E_{f} = \frac{1}{2}M_{*}\sigma_{0}^{2}-\frac{GM^{2}_{*}}{R_{0}} = -\frac{GM^{2}_{*}}{2R_{f}}
\end{equation*}
This implies that 
\begin{equation*}
    -\frac{M_{*}}{R_{f}} = \frac{1}{2}\frac{M_{g}+M_{*}}{R_{0}}-\frac{M_{*}}{R_{0}}
\end{equation*}
Finally, solving for $R_{f}$ yields
\begin{equation}
    \frac{R_{f}}{R_{i}} = \left[1-\frac{1}{2}\left(1+\frac{M_{g}}{M_{*}} \right) \right]^{-1} \quad R_{f}>0 \iff M_{g}<M_{*}
    \label{eq:infantmortality}
\end{equation}
Therefore, the cluster survives only if the expelled gaseous mass is smaller than the total mass in stars. This result is a strong lower limit to the maximum mass of gas which can be expelled without destroying the star cluster, because we assume instantaneous expulsion of the gas component. With more accurate calculations, assuming an actual timescale for the gas expulsion, we'd get $M_{g}\lesssim 4M_{*}$.
\subsection{Evaporation}
Evaporation of a cluster is yet another process able to disrupt a cluster. Describing the star cluster as a self-gravitating system, we can define an escape velocity for any given distance from its center 
\begin{equation*}
    v^{2}_{e}(r) = -2\phi(r)
\end{equation*}
The mean square escape velocity is then obtained by averaging over the (spherically symmetric) mass density distribution $\rho(r)$
\begin{equation*}
    \langle v^{2}_{e}\rangle = \frac{\int \rho(r)v^{2}_{e}(r)\,\text{d}r}{\int \rho(r)\,\text{d}r} = -\frac{2}{M}\int \rho(r)\phi(r)\,\text{d}r = -\frac{4E_{g}}{M}
\end{equation*}
where $E_{g}$ is the potential self-energy and $M$ is the total mass. Using the virial theorem once more leads to 
\begin{equation*}
    \langle v^{2}_{e}\rangle = 4\langle v^{2}\rangle
\end{equation*}
Therefore, stars with velocities exceeding twice the root-mean-square (RMS) velocity of the distribution are unbound. For a typical Maxwellian velocity distribution, this amounts to a fraction $\epsilon = 7.4\cdot 10^{-3}$ of all stars.
\par Roughly, evaporation removes $\text{d}N = -\epsilon N$ stars on a relaxation timescale, so the system gets slightly hotter and contracts. The velocity distribution adjusts to another Maxwellian distribution, so that in every relaxation time $\epsilon N$ stars are removed by evaporation
\begin{equation}
    \frac{\text{d}N}{\text{d}t} = -\frac{\epsilon N}{t_{rlx}} := -\frac{N}{t_{\text{evap}}}
    \label{eq:evaporation_ts}
\end{equation}
Note that the evaporation timescale is much greater than the relaxation timescale.
\subsection{Core Collapse of the Cluster}
As we're going to see, evaporation is usually enough to lead to \emph{core collapse}. In the last section we've shown that stars with velocity $v \geq v_{e} = 2\sqrt{\langle v^{2}\rangle}$ are unbound, adding up to the 0.74\% of the total population, if the velocity distribution were exactly Maxwellian, that is. Although evaporation is generally not a steady-state process, we can search for a self-similar solution that can describe evaporation with sufficient accuracy. In the self-similar regime, we expect a
constant rate of mass loss
\begin{equation*}
    \dot{M} = -\xi_{e}\frac{M(t)}{\tau_{rlx}(t)}
\end{equation*}
Neglecting changes in the logarithm in (\ref{eq:relaxation_time}), the relaxation time is proportional to 
\begin{equation*}
    \tau_{rlx} \propto \sqrt{M}R^{3/2}
\end{equation*}
so that 
\begin{equation*}
    \tau_{rlx} = \tau_{rlx}(0) \left(\frac{R(t)}{R(0)}\right)^{3/2}\left(\frac{M(t)}{M(0)}\right)^{1/2}
\end{equation*}
Using everything we've found so far, we get to 
\begin{equation*}
    \dot{M} = -\xi_{e}\frac{M(0)}{\tau_{rlx}(0)}\left(\frac{R(t)}{R(0)}\right)^{-3/2}\left(\frac{M(t)}{M(0)}\right)^{1/2}
\end{equation*}
Each star escaping from the cluster carries away a certain kinetic energy per unit mass 
\begin{equation*}
    \frac{\text{d}E_{tot}}{\text{d}M} = \zeta E_{m} = \zeta \frac{E_{tot}}{M}
\end{equation*}
As a consequence
\begin{equation*}
     \frac{\text{d}E_{tot}}{\text{d}t} =  \frac{\text{d}E_{tot}}{\text{d}M}\dot{M} = \zeta \frac{E_{tot}}{M}\dot{M}
\end{equation*}
But recall that $E_{tot}\propto -M^{2}/R$, so we have
\begin{equation*}
    \zeta \frac{E_{tot}}{M}\dot{M} \propto -\zeta \frac{M}{R}\dot{M}
\end{equation*}
On the other hand, we can also implicitly differentiate the relation for the total energy with respect to time
\begin{equation*}
    E_{tot}\propto -\frac{M^{2}}{R} \implies  \frac{\text{d}E_{tot}}{\text{d}t} \propto -\frac{2M}{R}\dot{M}+\frac{M^{2}}{R^{2}} \frac{\text{d}R}{\text{d}t}
\end{equation*}
In conclusion
\begin{equation*}
    (2-\zeta)\frac{\text{d}M}{M} = \frac{\text{d}R}{R}
\end{equation*}
This we can integrate right away
\begin{equation*}
    \frac{R(t)}{R(0)} = \left(\frac{M(t)}{M(0)}\right)^{2-\zeta}\implies \frac{\rho(t)}{\rho(0)}\propto \left(\frac{M(t)}{M(0)}\right)^{3\zeta-5}
\end{equation*}
For realistic clusters $\zeta < 1$; this makes sense since that most of the stars are ejected just barely over $v_{e}$, so that their velocity at infinity is generally $v_{\infty}<\sigma$. Since the typical energy per unit mass of particles is of order $\sigma^{2}$, we have $v^{2}_{\infty}<\sigma^{2}$ and thus $\zeta < 1$.
\par Note that $\rho(t)\to\infty$ as $M(t)\to 0$. We can now solve the equation for the mass flux $\dot{M}$, which bears as a result
\begin{equation*}
    M(t) = M(0)\left[1-\frac{\xi_{e}(7-3\xi)}{2}\frac{t}{\tau_{rlx}(0)}\right]^{2/(7-3\zeta)} := M(0)\left[1-\frac{t}{\tau_{0}}\right]^{2/(7-3\zeta)}
\end{equation*}
where we've defined the collapse time $\tau_{0}$ that satisfies 
\begin{equation*}
    M(\tau_{0}) = R(\tau_{0}) = 0
\end{equation*}
For a cluster composed of equal-mass stars, the collapse time is $\tau_{0}\geq 10\tau_{rlx}$.
\subsubsection*{Post Core-Collapse}
What happens after the core has collapsed due to the evaporation of the cluster?
\par We've seen that mass loss due to evaporation leads to collapse, a runaway process called gravothermal instability. If the system contracts, it becomes denser and the two-body encounter rate increases along with the evaporation rate. As a consequence, the core of the cluster loses energy (the kinetic energy of the evaporated stars) to the halo, and $\text{d}E_{\text{core}}<0$. Since for any bound, finite system in which the dominant force is gravity we have
\begin{equation*}
    C := \frac{\text{d}E}{\text{d}T} < 0
\end{equation*}
the temperature of the core must increase. Therefore, stars exchange more energy and become dynamically hotter, and faster stars tend to evaporate at a higher rate. This runaway process leads to the unphysical situation of star clusters with infinite core density.
\par To avoid this catastrophic scenario, we consider the possibility that the core collapse is reversed by an external source injecting kinetic energy into the core ($\text{d}E_{\text{core}} > 0)$, thus cooling it ($\text{d}T_{\text{core}} < 0$) until the temperature gradient declines to zero, thus halting the collapse. There are multiple ways this can happen: Mass loss by stellar winds and/or SNs; formation of binaries; three-body encounters between single stars and binaries extracting kinetic energy from the internal energy of the binary system and so on.
\par Consider this last scenario with the three stars initially far away from each other. The initial energy will then be just the sum of the respective kinetic energies $E = K_{1}+K_{2}+K_{3}$.
\par Once the binary is formed, the total energy will become $E = K_{\text{bin}}+E_{\text{bin}}+K'_{3}$. Since $E_{\text{bin}}<0$, conservation of energy implies
\begin{equation*}
    K_{\text{bin}}+K'_{3} > K_{1}+K_{2}+K_{3}
\end{equation*}
from which we see that the kinetic energy after the interaction is larger than the initial kinetic energy of the three stars. Therefore, the formation of binaries can pump kinetic energy into single stars crossing the core of the cluster (where most of the binaries form). Those stars then share the acquired extra kinetic energy with other stars through two-body relaxation, heating up the cluster.
\subsection{Mass Segregation and Spitzer's instability}
All we've gone through up until now works decently for any stellar population, given they have the same mass. In reality, not surprisingly, not all the stars in a cluster will have the same mass, and will rather sit somewhere into a spectrum ranging from $\sim 0.5M_{\odot}$ to $ \sim 150M_{\odot}$.
\par First of all, as we'll see in the next sections, massive-above-than-average stars are expected to go through a process called \emph{dynamical friction}. This means that a massive star walking through a sea of lighter stars feels a drag force, which decelerates its motion. The timescale of dynamical friction for a star of mass $M$ is approximately
\begin{equation*}
    \tau_{\text{segr}} = \frac{\langle m \rangle}{M_{tot}}\tau_{rlx}
\end{equation*}
where $\langle m \rangle$ is the average star mass and $\tau_{rlx}$ the two-body relaxation timescale. The effect of dynamical friction is that the most massive stars in a star cluster lose kinetic energy in favor of the light stars and segregate toward the centre of the star cluster. This generates the phenomenon called \emph{mass segregation}: The radial distribution of massive stars tends to be more centrally concentrated than the average stellar distribution in dense star clusters. Notice how such a process may be able to bring two stars close enough to form a binary system.
\par At equilibrium, energy is shared equally by all masses
\begin{equation}
    \frac{1}{2}m_{i}\langle v^{2}_{i} \rangle = \frac{1}{2}m_{j}\langle v^{2}_{j} \rangle
    \label{eq:equipartition_cond}
\end{equation} 
If the velocities of all stars are initially drawn from the same distribution, massive stars are thus expected to transfer kinetic energy to lighter stars and slow down, till they reach equipartition. But can equilbrium always be reached?
\par Consider the case of a cluster composed by $N_{1}$ stars of mass $m_{1}$ and $N_{2}$ of mass $m_{2}$ so that $m_{2}\gg m_{1}$ and $N_{2}\ll N_{1}$, in a way that, if we define $M_{i} = N_{i}m_{i}$, then $M_{2}\ll M_{1}$. Let $\rho_{i}$ be the local density of stars of mass $m_{i}$. From the virial theorem
\begin{equation*}
    \langle v^{2}_{i} \rangle = \alpha\frac{GM_{i}}{r_{i}}+\frac{G}{M_{i}}\int_{0}^{\infty}\rho_{i}\frac{M_{j}(r)}{r}4\pi r^{2}\,\text{d}r
\end{equation*}
where the first term describes the self-gravity of the population (with $\alpha$ a parameter that describes the density distribution throughout the cluster) and the second describes the gravitational energy that comes through the interaction of the two popolations; $r_{i}$ is the half-mass radius of the population.
\par As a consequence of segregation, the more massive stars become centrally concentrated compared to the distribution of low-mass stars. We can thus assume that the density of lighter stars is constant $\rho_{1}(r)\approx\rho_{1}(0)$ throughout the region occupied by the heavier stars, so that 
\begin{equation*}
    M_{1}(r) = \frac{4\pi}{3}r^{3}\rho_{cl,1}
\end{equation*}
where $\rho_{cl,1}$ is the central density of stars of mass $m_{1}$. Under these assumptions, the equations for the RMS-velocities are simplified\footnote{In the equation for $\langle v^{2}_{2} \rangle$ we substitute the relation for $M_{1}$ we've written above.}
\begin{align*}
    \langle v^{2}_{1} \rangle &= \frac{\alpha GM_{1}}{r_{1}}\\
    \langle v^{2}_{2} \rangle &= \frac{\alpha GM_{2}}{r_{2}}+\frac{4\pi G}{3}\rho_{cl,1}R_{sp,2}^{2}
\end{align*}
where we've defined Spitzer's radius as
\begin{equation}
    R_{sp,2}^{2} = \frac{1}{M_{2}}\int_{0}^{\infty}r^{2}\rho_{2}(4\pi r^{2})\,\text{d}r
    \label{eq:spitzerradius}
\end{equation}
Defining the mean density of stars of each type within their half-mass radius
\begin{equation*}
    \rho_{m,i} = \frac{3}{4\pi r^{3}_{i}}M_{i}
\end{equation*}
we can invert it and express the half-mass radius in terms of the mean density of the $i$-th population, so that substituting into the equipartition condition (\ref{eq:equipartition_cond}) yields
\begin{equation}
    \chi = \frac{M_{2}}{M_{1}}\left(\frac{m_{2}}{m_{1}}\right)^{\frac{3}{2}} = \left(\frac{\rho_{m,1}}{\rho_{m,2}}\right)^{\frac{1}{2}}\left[1+\beta\left(\frac{\rho_{m,1}}{\rho_{m,2}}\right)\right]^{-\frac{3}{2}} \quad \beta = \frac{\rho_{c,1}}{\rho_{c,2}}\frac{1}{2\alpha}\left(\frac{R_{sp,1}}{R_{sp,2}}\right)^{2}
    \label{eq:spitzer_gen}
\end{equation}
The expression above has the maximum value
\begin{equation*}
    \frac{\rho_{m,1}}{\rho_{m,2}} = (2\beta)^{-1} \implies \chi_{\text{max}} = \sqrt{\frac{4}{27\beta}}
\end{equation*}
A realistic value for $\beta$ is $\approx 5.8$, so that $\chi_{\text{max}}\approx 0.16$. So, Spitzer's condition (\ref{eq:spitzer_gen}) predicts equilibrium against mass segregation only if 
\begin{equation}
    \frac{M_{2}}{M_{1}}\left(\frac{m_{2}}{m_{1}}\right)^{3/2} < 0.16
    \label{eq:spitzer_cond}
\end{equation}
For a realistic cluster, $m_{2}\approx 10m_{1}$ and $M_{2}\approx 0.03M_{1}$. This implies $\chi \approx 1 > \chi_{\text{max}}$, so there's no reaching equilibrium and massive stars are dragged towards the center as a result of mass segregation. This system will thus be prone to the formation of tight SBH binaries via capture and other processes.
\section{Stellar and BH Binaries Hardening}
Recall that the internal energy of a binary is 
\begin{equation*}
    E_{\text{bin}} = -\frac{GM_{1}M_{2}}{2a} = -E_{b}
\end{equation*}
Consider now a 3-body interaction between the binary and a third object $m_{3}$ (a "intruder"). If the original binary is preserved in the encounter, there are two possibilities
\begin{itemize}
    \item the single body extracts internal energy from the binary, so that the final kinetic energy of center of mass of the intruder and of the binary is higher than the initial one;
    \item the single body loses a fraction of its kinetic energy, which is converted into internal energy of the binary.
\end{itemize}
In the first case, the object and the binary acquire recoil velocity and the binding energy increases
\begin{equation*}
    K_{i}-E_{b,i} = K_{f}-E_{b,f} \implies E_{b,f}-E_{b,i} = K_{f}-K_{i}
\end{equation*}
for $K_{f}>K_{i}$\footnote{Note that this are the total kinetic energies of the 3-body system.}
\begin{equation*}
    E_{b,f} = \frac{GM_{1}M_{2}}{2a_{f}} > \frac{GM_{1}M_{2}}{2a_{i}} = E_{b,i} \implies a_{f}<a_{i}
\end{equation*}
The result is flipped in the other case.
\par Another possibility for the binary to increase the binding energy during a 3-body interaction is through "exchange", for example if the intruder replaces one of the members of the binary. This usually happens when $M_{2}<m_{3}<M_{1}$, in which case, after the exchange the binary is formed by $M_{1}$ and $m_{3}$
\begin{equation*}
    E_{b,f} = \frac{GM_{1}m_{3}}{2a} > \frac{GM_{1}M_{2}}{2a} = E_{b,i}
\end{equation*}
The final binary can also becomes less bound and can even be "ionized" if its velocity at infinity exceeds the critical velocity $v_{c}$. In fact
\begin{equation*}
    E_{f} = \frac{1}{2}\frac{m_{3}(M_{1}+ M_{2})}{(M_{1}+M_{2}+m_{3})}v^{2}-\frac{GM_{1}M_{2}}{2a}
\end{equation*}
and the system is unbound if $E_{f}=0$, so if
\begin{equation}
    v_{c} = \sqrt{\frac{GM_{1}M_{2}(M_{1}+M_{2}+m_{3})}{am_{3}(M_{1}+M_{2})}}
    \label{eq:critical_velocity}
\end{equation}
We define \emph{hard binaries} those with a binding energy greater than $\langle m \rangle \sigma^{2}/2$, where $\sigma$ is the average velocity of the stars and $\langle m \rangle$ the average mass. Conversely, \emph{soft binaries} satisfies the inverse relation.
\par \emph{Heggie's law} states that hard (soft) binaries only tend to get harder (softer).
\subsubsection*{Cross Section for 3-body Encounters}
To define the cross section for 3-body encounters, let us consider the maximum impact parameter $b_{\text{max}}$ for a non-zero energy exchange between the single object $m_{3}$ and the binary. To estimate the impact parameter, we need to consider gravitational focusing, that is the fact that the trajectory of the intruder is significantly deflected by the presence of the binary, thus approaching it with an effective pericentre $p$ much smaller than the formal impact parameter $b$ at infinity.
\par Conservation of energy implies 
\begin{equation*}
    \Delta E = 0 = \frac{m_{3}(M_{1}+M_{2})}{M_{1}+M_{2}+m_{3}}(v^{2}_{f}-v^{2}_{i})+Gm_{3}(M_{1}+M_{2})\left(\frac{1}{D}-\frac{1}{p}\right)
\end{equation*}
where $D$ is the initial distance between the single object and the binary, and for the initial velocity of the single object we consider $v_{i} = \sigma$. Assuming $D\gg p$ (the periastron), the equation simplifies to 
\begin{equation*}
    \frac{1}{2}\frac{\sigma^{2}}{M_{1}+M_{2}+m_{3}} = \frac{1}{2}\frac{v^{2}_{f}}{M_{1}+M_{2}+m_{3}}-\frac{G}{p}
\end{equation*}
On the other hand, we also have to consider angular momentum conservation
\begin{equation*}
    \Delta J = 0 = (pv_{f}-b\sigma)\frac{m_{3}(M_{1}+ M_{2})}{(M_{1}+M_{2}+m_{3})}
\end{equation*}
so that $pv_{f}=b\sigma$. Combining the two conservation equations and solving for $p$
\begin{equation*}
    p = \frac{G(M_{1}+M_{2}+m_{3})}{\sigma^{2}}\left[\sqrt{1+\frac{b^{2}\sigma^{4}}{G^{2}(M_{1}+M_{2}+M_{3})^{2}}}-1\right]
\end{equation*}
which can be Taylor expanded for 
\begin{equation*}
    \frac{b^{2}\sigma^{4}}{G^{2}(M_{1}+M_{2}+M_{3})^{2}} \ll 1
\end{equation*}
to finally yield 
\begin{equation}
    p \simeq \frac{b^{2}\sigma^{2}}{2G(M_{1}+M_{2}+M_{3})}
    \label{eq:periastron}
\end{equation}
The 3-body cross section is then defined just as 
\begin{equation*}
    \Sigma = \pi b^{2}_{\text{max}} \simeq \pi \left[\frac{2G(M_{1}+M_{2}+m_{3})}{\sigma^{2}}\right]p_{max} \simeq \frac{2\pi G(M_{1}+M_{2}+m_{3})a}{\sigma^{2}}
\end{equation*}
where we have approximated $p_{\text{max}}\simeq a$, which holds only for very energetic 3-body encounters.
\subsection{3-body Hardening}
Since we now have an expression for the 3-body cross section, we can estimate the interaction rate
\begin{equation*}
    \frac{\text{d}N}{\text{d}t} = n\Sigma\sigma = \frac{2\pi G(M_{1}+M_{2}+m_{3})na}{\sigma}
\end{equation*}
We now make a series of simplifying assumptions that characterize those binaries that will eventually become GW sources. Importantly, those assumptions are
relevant for SBHs and SMBHs alike, thus providing a useful description to the dynamics of SBH binaries and MBH binaries. We assume that
\begin{enumerate}
    \item the binary is hard;
    \item the effective pericentre satisfies $p \lesssim 2a$;
    \item the mass of the intruder is small in respect to the binary $m_{3}\ll M_{1},\,M_{2}$.
\end{enumerate}
This way, the average binding energy variation per encounter reads something like 
\begin{equation*}
    \langle \Delta E_{b} \rangle = \xi \frac{m_{3}}{M_{1}+M_{2}}E_{b} = \xi\frac{m_{3}}{M_{1}+M_{2}}\frac{GM_{1}M_{2}}{2a}
\end{equation*}
where $\xi \approx 0.2-1$ is a parameter that can be extracted from 3-body scattering experiments. The rate of binding energy exchange for a hard binary is
\begin{equation*}
    \frac{\text{d}E_{b}}{\text{d}t} = \langle \Delta E_{b} \rangle \frac{\text{d}N}{\text{d}t} = 2\pi\xi \frac{M_{1}M_{2}m_{3}(M_{1}+M_{2}+m_{3})}{M_{1}+M_{2}}\frac{G^{2}n}{\sigma}
\end{equation*}
Supposing a single mass population of intruders
characterized by $m_{3} = \langle m \rangle$, we can write the rate of binding energy exchange in terms of the local mass density $\rho = n\langle m \rangle$. Exploiting the mass "hierarchy" condition
\begin{equation*}
    \frac{\text{d}E_{b}}{\text{d}t} = \frac{2\pi\xi G^{2}M_{1}M_{2}\rho}{\sigma}
\end{equation*}
Therefore, hard binaries harden at a constant rate!
\par Expressing $a$ in terms of $E_{b}$ the hardening rate is given by 
\begin{equation*}
    \frac{\text{d}}{\text{d}t}\left(\frac{1}{a}\right) = \frac{2}{GM_{1}M_{2}}\frac{\text{d}E_{b}}{\text{d}t} = 4\pi G\xi\frac{\rho}{\sigma} = \frac{GH\rho}{\sigma^{2}}
\end{equation*}
which can be written as
\begin{equation}
      \frac{\text{d}a}{\text{d}t} = -\frac{GH\rho}{\sigma}a^{2}
      \label{eq:hardening_rate}
\end{equation}
where we've defined $H\approx 15-20$ as a dimensionless hardening rate\footnote{The lower bound is for circular orbits, while 20 is for very eccentric orbits.}.
\subsection{Hardening and GWs}
How is all this stuff connected to GWs?
\par From (\ref{eq:hardening_rate}) we can see that hardening in a given stellar background proceeds at a constant rate determined by the properties of the background. The evolution of semi-major axis can be thought as composed of two contributions
\begin{equation*}
    \frac{\text{d}a}{\text{d}t} = \left(\frac{\text{d}a}{\text{d}t}\right)_{3b}+ \left(\frac{\text{d}a}{\text{d}t}\right)_{GW} = -Aa^{2}-Ba^{-3}
\end{equation*}
where we have expliticted the dependence on $a$ of both terms and defined
\begin{equation*}
    A = \frac{GH\rho}{\sigma} \quad B = \frac{64}{5}\frac{G^{3}}{c^{5}}M_{1}M_{2}(M_{1}+M_{2})F(e)
\end{equation*}
Since stellar hardening is $\propto a^{2}$ and the GW hardening is $\propto a^{-3}$, binaries spend most of their time at the transition separation obtained by imposing $(\text{d}a/\text{d}t)_{3b} = (\text{d}a/\text{d}t)_{GW}$
\begin{equation*}
    \bar{a} = \left[\frac{64G^{2}\sigma M_{1}M_{2}(M_{1}+M_{2})F(e)}{5c^{5}H\rho}\right]^{1/5}
\end{equation*} 
and their lifetime can be written as 
\begin{equation*}
    \tau(\bar{a}) =\frac{\sigma}{GH\rho\bar{a}} \approx 3\,\text{Gyr}\left(\frac{\sigma}{10\,\text{km}\,\text{s}^{-1}}\right)^{1/5}\left(\frac{\rho}{10^{5}M_{\odot}\,\text{pc}^{-3}}\frac{\bar{a}}{0.15\,\text{AU}}\right)^{-1}
\end{equation*}
Given this expression, is it possible to enter the GW hardening regime in less than a Hubble time? Yes!
\par For typical\footnote{Typical as long as massive SBH binaries are concerned.} values, $M_{1}+M_{2}\approx 60M_{\odot}$, $\sigma = 10\,\text{km}\,\text{s}^{-1}$, $\rho = 10^{5}M_{\odot}\,\text{pc}^{-3}$, you can get a hardening time $\tau(\bar{a})\approx 3\,\text{Gyr}<t_{\text{Hubble}}$.
\par Moreover, the mechanism efficiency increases with the BH binary mass\footnote{Recall that $\bar{a} \propto M_{1}M_{2}(M_{1}+M_{2})$, while $\tau(\bar{a})\propto \bar{a}^{-1}$.}. If IMBHs can indeed form in star clusters, stellar hardening provides an efficient mechanism to merge them with SBHs or with a companion IMBH. 
\section{Supermassive Black Holes}
We now turn to discuss some relevant astrophysical aspects of (S)MBHs. Those objects has been observed at the centre of massive galaxies, and inhabit virtually all nuclei of galaxies with $M_{*} > 10^{11}M_{\odot}$, whereas their ubiquity in lighter galaxies is much debated. Our discussion will be mainly classical (Newtonian), although some results of GR will be implemented when necessary.
\subsection{Bondi accretion}
In order to accrete, a MBH needs to capture gas from its surroundings at a sufficient rate. We've already discussed about this in Chapter \ref{ch:vi}, but a brief refresher won't do much harm.
\par The model assumes an object (a black hole in our case) of mass $M$ surrounded by an infinite cloud of gas, accreting with stationary and spherically
symmetric motion. The model neglects any self-gravity effects of the cloud, magnetic fields, angular momentum and viscosity due to the accretion mechanism.
\par The gas is assumed to be perfect and polytropic
\begin{equation*}
    p = p_{\infty}\left(\frac{\rho}{\rho_{\infty}}\right)^{\gamma}
\end{equation*}
with $\gamma$ the polytropic index. Because of stationarity and spherical symmetry, conservation of mass $D_{\mu}(\rho u^{\mu}) = 0$ implies a constant accretion rate
\begin{equation*}
    \dot{M} = 4\pi r^{2}\rho v = \text{const.}
\end{equation*}
where $r$ is the radial coordinate and $v$ in the inward velocity of the gas. We can also define the sound speed of the gas $c^{2}_{s} = \gamma p_{\infty}/\rho_{\infty}$ at infinity and a characteristic lengthscale of the problem, called \emph{Bondi radius}
\begin{equation*}
    r_{B} = \frac{GM}{c^{2}_{s}}
\end{equation*}
I hope all these things are actually ringing some bells, no matter how out of tune they may be. We can the rescale the variables
\begin{align*}
    r &= xr_{B}\\
    v &= yc_{s}\\
    \rho &= z\rho_{\infty}
\end{align*}
All the dynamic of the system can be thus summarized in just two equations
\begin{align*}
    x^{2}yz &= \lambda \\
    \frac{y^{2}}{2} + \frac{z^{\gamma-1}-1}{\gamma -1}-\frac{1}{x} &= 0
\end{align*}
where the dimensionless accretion rate parameter $\lambda$ is just
\begin{equation*}
    \lambda = \frac{\dot{M}_{B}}{4\pi r^{2}_{B}c_{s}\rho_{\infty}} = \frac{\dot{M}c^{3}_{s}}{4\pi G^{2}M^{2}\rho_{\infty}}
\end{equation*}
For $\lambda = 1$, we get Bondi's accretion rate
\begin{equation*}
    \dot{M}_{B} \approx 2.5\cdot 10^{2}\left(\frac{c_{s}}{100\,\text{km}\,\text{s}^{-1}}\right)^{-3}\left(\frac{M}{10^{8}M_{\odot}}\right)^{2}M_{\odot}\,\text{yr}^{-1}
\end{equation*}
Therefore $\dot{M}_{B}\propto c^{-3}_{s}M^{2}$. The mass dependence, in particular, implies that the mass of the compact object diverges to infinity in a finite amount of time for an infinite "fuel" supply. \par It should not surprise you that this is largely unphysical, for once we're assuming \emph{perfect} spherical symmetry and neglecting all feedbacks due to the radiation emitted by the accretion flow.
\subsection{The Eddington Limit}
Once again, here's something we've already got acquainted with back in Chapter \ref{ch:vi}. 
\par The radiation reaction onto the accretion flow is the physical rationale behind the Eddington accretion limit, and sets the maximum luminosity $L$ that an AGN
(or, in fact, any astrophysical object) can emit when the radiation force acting outward equals the gravitational force acting inward. Beyond this limit, the radiation force overwhelms the gravitational force and the accretion process is considerably softened or halted.
\par Let's assume a spherically symmetric cloud of fully ionized Hydrogen around our BH, so that the main channel of interaction for photons is the Thomson scattering.
\par The flux of energy through a spherical surface of radius $r$ is 
\begin{equation*}
    \Phi = \frac{L}{4\pi r^{2}}
\end{equation*}
and the momentum flux (pressure) is
\begin{equation*}
    p_{\text{rad}} = \frac{\Phi}{c} = \frac{L/c}{4\pi r^{2}}
\end{equation*}
Therefore, the force exerted by the radiation on a single electron is given by
\begin{equation*}
    F_{\text{rad}} = p_{\text{rad}}\sigma_{Th} = \frac{L}{4\pi r^{2}c}\frac{8\pi}{3}\left(\frac{e^{2}}{m_{e}c^{2}}\right)^{2}
\end{equation*}
The dependence of the cross section on the particle mass justifies our assumption to neglect scattering with heavier particles, namely protons.
\par We can write the gravitational force too
\begin{equation*}
    F_{g} = \frac{GM(m_{e}+m_{p})}{r^{2}}\approx \frac{GMm_{p}}{r^{2}}
\end{equation*}
Equating the two forces and solving for $L$ gives us our longed for Eddington's luminosity
\begin{equation}
    L_{\text{Edd}} = \frac{3GMm_{p}c}{2}\left(\frac{m_{e}c^{2}}{e^{2}}\right)^{2}\approx 1.26\cdot 10^{28}\left(\frac{M}{M_{\odot}}\right)\,\text{erg}\,\text{s}^{-1}
    \label{eq:eddingtonluminosity}
\end{equation}
Now, assume that the accretion process occurs at a given rate $\dot{M}$, and that only a fraction\footnote{The fraction $\epsilon$ is the radiative efficiency of the accretion process and can be estimated from the energy loss of the accreted material.} $\epsilon$ of the rest mass energy of the accreted matter is radiated away. Then, the luminosity can be expressed as
\begin{equation*}
    L = \epsilon \dot{M} c^{2}
\end{equation*}
thus setting a limit on the accretion rate
\begin{equation*}
    \dot{M}_{\text{Edd}} = \frac{4\pi GMm_{p}}{c\sigma_{th}}\frac{M}{\epsilon}\approx 2.2\cdot 10^{-8}\left(\frac{\epsilon}{0.1}\right)\left(\frac{M}{M_{\odot}}\right)M_{\odot}\,\text{yr}^{-1}
\end{equation*}
For a gas element in a wide initial orbit, we can safely assume $E_{i} = 0$, while upon reaching the inner rim of the accretion disk in an (approximately) circular orbit, its energy is given by $E_{\text{rim}}=-GMm/2r_{\text{rim}}$. We can then evaluate the luminosity of the disk as
\begin{equation*}
    L = -\frac{\text{d}E}{\text{d}t} = \frac{G\dot{m}M}{2r_{\text{rim}}}
\end{equation*}
where $\dot{m} = -\text{d}m/\text{d}t$ it the mass accretion rate. At this point we can identify $r_{\text{rim}}$ with the ISCO (Innermost Stable Circular Orbit), whose expression is a standard GR result\footnote{At least for Schwartzschild BHs you can try calculate $r_{\text{ISCO}}$ as an exercise. The reader too lazy to do it him/herself can see \cite{carroll}, §5 and §6.}
\begin{equation*}
    r_{\text{rim}} = r_{\text{ISCO}} = \beta\frac{2GM}{c^{2}} = \beta R_{s}
\end{equation*}
so that
\begin{equation*}
    L = \frac{1}{4\beta} \dot{M}c^{2} = \epsilon \dot{M}c^{2} \implies \epsilon = 4\beta
\end{equation*}
The value of $\beta$ depends on the particular specimen of BH: For Schwartzschild's BHs $\beta = 3$, while for Kerr's extremal (maximally spinning) BHs $\beta = 1$.
\subsection{MBH Growth}
Our most recent understanding of MBHs involves the existence of BH \emph{seeds} at high redshift that had become supermassive as they accreted gas, stars and possibly even merged with other compact objects.
\par The need to start from seeds with $M<10^{6}M_{\odot}$ is dictated by the fact that no physical mechanism able to monolithically form a billion stellar-mass compact object is known at present day.
\par From Eddington's luminosity (\ref{eq:eddingtonluminosity}) we can define Eddington's timescale
\begin{equation}
    \tau_{\text{Edd}} = \frac{Mc^{2}}{L_{\text{Edd}}} = \frac{\sigma_{th}c}{4\pi Gm_{p}} \approx 0.45\,\text{Gyr}
    \label{eq:eddington_timescale}
\end{equation}
We can then write down the evolution equation for the mass of the MBH as
\begin{equation*}
    \dot{M} = (1-\epsilon)\dot{M}_{acc} = \frac{1-\epsilon}{\epsilon}\frac{f_{\text{Edd}}}{\tau_{\text{Edd}}}M
\end{equation*}
where $f_{\text{Edd}} = L/L_{\text{Edd}}$ is the fraction of the Eddington luminosity being radiated. The last equation can be integrated right away
\begin{equation*}
    M(t) = M_{0}\exp\left(\frac{1-\epsilon}{\epsilon}\frac{f_{\text{Edd}}}{\tau_{\text{Edd}}}(t-t_{0})\right)
\end{equation*}
where $M_{0}$ is the initial mass of the MBH at time $t_{0}$. In the limit $f_{\text{Edd}}\to 1$, 
\begin{equation*}
    M(t) = M_{0}\exp\left(\frac{1-\epsilon}{\epsilon}\frac{t-t_{0}}{\tau_{\text{Edd}}}\right)
\end{equation*}
we can now see what is the fastest rate at which a MBH can grow. This evidently depends on the efficiency parameter $\epsilon$, which, we've seen, is closely related to the spin parameter of the BH $\beta$
\begin{itemize}
    \item $\epsilon = 1/12$ (S) $\to M \propto M_{0}\exp(\frac{t}{3\cdot 10^{7}\,\text{yr}})$
    \item $\epsilon = 1/4$ (K) $\to M \propto M_{0}\exp(\frac{t}{3\cdot 10^{8}\,\text{yr}})$
\end{itemize}
Therefore, in both scenarios, continuous Eddington-limited accretion allows 50-to-500 $e$-folds in mass, depending on the BH's spin: This is more than enough to grow a $10^{9}M_{\odot}$ BH starting from any reasonable $M_{0}$ in less than a Hubble time.
\subsubsection*{Seeding Mechanisms}
We've shown how, given the existence of a seed, we're able to accrete a BH up to a MBH, but we've never discussed about what actually are these seeds we're talking about.
\par In a very Hegelian way\footnote{Or PokéMon-ian kind of way, if you're more inclined towards a modern time comparison.}, seeds mainly comes in three "flavors":
\begin{itemize}
    \item \emph{PopIII Remnant}: This scenario relies on the fact that the first generation of essentially metal free stars is expected to have a very top-heavy mass function. As we've seen, the absence of metal disfavors fragmentation and massive stars can form. The natural relics of such massive stars are BHs of several hundred solar masses. The viability of popIII remnant as seeds of the most massive MBHs has been recently questioned;
    \item \emph{Runaway PopIII Mergers}: If clusters of massive stars are a common occurrence at high redshifts, runaway mergers might still result in the formation of a metal free star with mass high enough to leave behind a $10^{3}M_{\odot}$ BH remnant;
    \item \emph{Direct Collapse}: The key idea is that in the most massive protogalactic halos at $z \approx 15$, gas accreted from the cosmic web can be supplied to the very centre at a rate of $\approx 1M_{\odot}\,\text{yr}^{-1}$. Such extreme conditions can prompt the formation of a seed BH with mass in the range $10^{4}-10^{5}M_{\odot}$ either via collective infall from a marginally stable massive disk or via the formation of a quasi star or via direct collapse.
\end{itemize}
\par One challenge to this simple picture is that billion-solar-masses MBHs are not observed only in the local Universe (i.e., low redshift), but also at high redshift, which issues serious questions since they would have had really short times to accrete to such huge masses (assuming the model we've proposed is actually correct).
\par This is a problem, because prolonged accretion inevitably results in highly spinning MBHs after about an $e$-fold in
mass growth. It seems therefore unlikely that the highest redshift quasars grew by Eddington limited, prolonged accretion, independently on the seeding model.
\par Now, this problem is somewhat mitigated with an argument the interested reader can find in \cite{celoria2018lecturenotesblackhole} (pag. 59).
\subsection{Massive BH Binaries}
We start this new subsection with a question: Can Supermassive BHs really merge?
\begin{figure}[h!]
    \centering 
    \includegraphics[width=0.8\textwidth]{img/smbhb.png}
    \caption{Different phases of the evolution of a MBH binary after a galaxy merger. Credits: \cite{celoria2018lecturenotesblackhole}}
    \label{fig:smbhb}
\end{figure}
\par Turns out, they probably can. Despite growing most of their mass through accretion along the cosmic history, in the hierarchical structure formation framework MBHs will also acquire some mass because of mergers with other MBHs. Galaxies are in fact observed to merge quite regularly, with massive galaxies experiencing at least a major merger. If each of the progenitor galaxies host a MBH, the outcome of a galaxy merger will be the formation of a MBH binary that eventually coalesces due to GW emission.
\par The evolution of the system is pictorially represented in Fig.\ref{fig:smbhb}.
\par We already had a look at some of the processes that can reduce the distance under the pc separation and at GW emission, that leads to swift coalescence at sub-parsec sepratation. What we're missing is a process able to bring two BHs from kpc to pc separation. An effective process to do so is \emph{dynamical friction}, and we're going to take a look at it in the next (and last) subsection.
\subsection{Dynamical Friction}
Consider a lonely massive object of mass $M$ and velocity $V$ wandering in a sea of particles of mass $m$ and velocity $v$\footnote{The velocity of the massive object is to be considered with respect to the Center of Mass of the sea of particles.}. The idea is having these small particles interacting with the massive intruder through a collective force that goes under the name of \emph{dynamical friction}.
\par A first detailed calculation was carried out by Chandrasekhar
\begin{equation}
    \frac{\text{d}\vec{V}}{\text{d}t} = -16\pi^{2}\ln(\Lambda)G^{2}m(M+m)\frac{\vec{V}}{V^{3}}\int_{0}^{V}f(v)v^{2}\,\text{d}v
    \label{eq:chandrasekhar_eq}
\end{equation}
where $f(v)$ is the velocity distribution function of the light particles and $\ln\Lambda$ is the Coulomb logarithm $\Lambda = b_{\text{max}}/b_{\text{min}}$.
\par In the limit $V \to 0$, we can approximate $f(v)$ roughly as constant, and the expression in (\ref{eq:chandrasekhar_eq}) evaluates to
\begin{equation*}
     \frac{\text{d}\vec{V}}{\text{d}t} = -\frac{16\pi^{2}}{3}\ln(\Lambda)G^{2}m(M+m)f_{0}\vec{V}
\end{equation*}
which has the form typical of a viscous friction $\text{d}V/\text{d}t \propto -V$.
\par Conversely, in the limit $V \gg \bar{v}$, with $\bar{v}$ the typical velocity of the distribution, the integral is performed over the whole distribution, thus returning the number density $n/4\pi$. Thus
\begin{equation*}
    \frac{\text{d}\vec{V}}{\text{d}t} = -4\pi\ln(\Lambda)G^{2}M\rho\frac{\vec{V}}{V^{3}}
\end{equation*}
where we've made use of $\rho = nm$ and $M \gg m$. At high velocities we see that $\text{d}V/\text{d}t \propto -V^{-2}$ and DF quickly becomes ineffective.
\par Note that, due to the proportionality with $M$, the force $F \propto M^{2}$; this is usually interpreted as the pull of a wake formed by DF behind the massive object.
\par Given these two limits, it's evident that DF is more efficient when $V \approx \bar{v}$.
\par Let us now consider the ideal situation of an isothermal sphere of particles. If we assume the massive object in circular orbit, we can calculate the frictional force by explicitly solving Chandrasekhar's equation (\ref{eq:chandrasekhar_eq})\footnote{We're assuming a Maxwellian distribution $f(v)$.}, finding
\begin{equation*}
    F \approx -0.428 \ln(\Lambda)\frac{GM^{2}}{r^{2}}
\end{equation*}
The angular momentum of the object moving in circular motion is simply $J = Vr$, so that
\begin{equation*}
    \frac{\text{d}J}{\text{d}t} = \frac{\text{d}V}{\text{d}t}r + V\frac{\text{d}r}{\text{d}t}
\end{equation*}
Now, we're going to perform what may look like a magic trick. Since, instantaneously, DF does not act on $r$ and only changes the velocity of the massive object, we can write $\text{d}J/\text{d}t = (\text{d}V/\text{d}t)r = (F/M)r$. On the other hand, the massive object is in circular orbit in an isothermal sphere. Under the approximation that the orbit remains circular, $V$ cannot change so that the actual result of the interaction would be to move the object onto a tighter orbit, thus shrinking $r$. In practice, DF does not change the kinetic energy of $M$, but it eventually extracts its potential energy,
so that we can write
\begin{equation*}
     \frac{\text{d}J}{\text{d}t} = \frac{F}{M}r = V \frac{\text{d}r}{\text{d}t}
\end{equation*}
We can plug in the expression for the force we've found earlier 
\begin{equation*}
    V \frac{\text{d}r}{\text{d}t} = -0.428 \ln(\Lambda)\frac{GM^{2}}{r^{2}}
\end{equation*}
that we can integrate by parts to get
\begin{equation}
    \tau_{f} = \frac{1.17}{\ln\Lambda}\frac{r^{2}_{i}v_{c}}{GM} = \frac{19}{\ln\Lambda}\,\text{Gyr}\left(\frac{r_{i}}{5\,\text{kpc}}\right)^{2}\frac{\sigma}{200\,\text{km}\,\text{s}^{-1}}\frac{10^{8}M_{\odot}}{M}
    \label{eq:df_timescale}
\end{equation}
where $r_{i}$ is an integration constant we can identify with some kind of initial distance. For typical $\ln\Lambda\approx 10-15$, MBHs can inspiral in the centre of the stellar remnant from a $10\,\text{kpc}$ initial distance in less than a Gyr (which is less than a Hubble time!).
\par This is the time it takes to bring a single MBH to the centre of an isothermal distribution of stars. It can be applied to the two MBHs inspiralling in the aftermath of a galaxy merger so long as they evolve independently of each other as individual objects interacting with the stellar distribution. This is no longer true when the two MBHs start to "see each other."
\par For an isothermal sphere this amounts to a mutual separation of 
\begin{equation*}
    a \approx 30\,\text{pc} \left(\frac{M}{10^{8}M_{\odot}}\right)^{1/2}
\end{equation*}
Once the two BHs overlap with the cloud's center of mass, DF cannot bring them any closer. To do so, we need Stellar Hardening and all the other nice processes we've discussed earlier.
\begin{flushright}
    \emph{The End?}
\end{flushright}