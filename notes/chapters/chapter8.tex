\chapter{Black Holes Binaries and Coalescence}
\section{Introduction}
This last part of the notes will focus on (Astrophysical) Black Holes, going through the formation of BH binaries and all the processes that make possible the formation of such systems.
\par I'll closely follow my own notes, as well as those made available by Professor M. Crisostomi \cite{celoria2018lecturenotesblackhole}. However, I won't fail to suggest books and/or article for those who may be interested.
\vspace{0.5cm}
\par It is common to identify astrophysical black holes according to their mass
\begin{itemize}
    \item \emph{Stellar-mass Black Holes} (SBHs); they have masses in the range $M_{\odot}\lesssim M \lesssim 10^{2} M_{\odot}$. SBHs are what remains of stars with initial mass of the order $M\gtrsim 20M_{\odot}$.
    \item \emph{Intermediate-mass Black Holes} (IMBHs); they have masses in the range of $10^{2}M_{\odot}\lesssim M \lesssim 10^{6}M_{\odot}$. Since we have only scarce observative evidence about these guys, they're the ones we know the least about. We know that they can form at high redshift\footnote{See the next subsection for a short debrief about redshift.} either from PopIII star remnants\footnote{"Population III stars are a hypothetical population of extremely massive, luminous and hot stars with virtually no "metals", except possibly for intermixing ejecta from other nearby, early population III supernovae. Such stars are likely to have existed in the very early universe (i.e., at high redshift) and may have started the production of chemical elements heavier than hydrogen, which are needed for the later formation of planets and life as we know it." Definition from \href{https://en.wikipedia.org/wiki/Stellar_population}{Wikipedia}.} or via direct collapse from a marginally stable protogalactic disc or quasistar, or at lower redshifts via dynamical processes in massive star clusters.
    \item \emph{(Super) Massive Black Holes} (MBHs or SMBHs); they have masses in the range of $10^{6}M_{\odot}\lesssim M \lesssim 10^{10}M_{\odot}$. They are observed at the center of almost all galaxies, AGNs and Quasars.
\end{itemize}
\begin{figure}[h!]
    \centering 
    \includegraphics[width=0.8\textwidth]{img/bh_massdistrib.png}
    \caption{Pictorial representation of the mass ranges of known black holes.\\ Credits: \cite{Colpi_2017}.}
\end{figure}
\par Most of the objects of astrophysical interest (stars, neutron stars, BHS etc.) are typically observed in binary systems. For example, half of the total amount of stars and, more interesting, 70\% of the stars with mass greater than ten solar masses, that are relevant to the formation of compact objects, are in binaries. On top of that, as shown in any GR course, two compact objects bound in a binary system have a non-zero quadrupole moment and are thus able to emit GWs.
\par After a short digression about redshift, more for the sake of completeness than anything, we'll start turning our attention to the physical features of binary systems. In particular, we'll consider the Keplerian motion of the binary, where two compact objects can be regarded as point masses and, as long as the orbital velocities are small compared to the speed of light.
\par In short, unless pointed out otherwise, our discussion will be mainly classical.
\subsubsection{A short detour on redshift}
In our description of the Universe we'd like to determine a number of quantities observationally to try to infer which of the models proposed by the FRW metric\footnote{As per any GR subject in this notes, I'll follow \cite{carroll}. You can find this discussion at §8.5.}
\begin{equation}
    \text{d}s^{2} = -\text{d}t^{2}+a(t)^{2}\left(\frac{\text{d}r^{2}}{1-kr^{2}}+r^{2}\text{d}\Omega_{(2)}^{2}\right)
    \label{eq:FRW}
\end{equation}
corresponds to our Universe. First and foremost, we could perform a slight manipulation of the FRW metric (\ref{eq:FRW}) and define a conformal time 
\begin{equation*}
    a(\eta)\text{d}\eta = \text{d}t
\end{equation*}
so that the metric is, guess what, conformal. The scale factor $a$ is therefore factorized and negligible if we consider null-like trajectories $\text{d}s^{2} = 0$. Note that in 
\begin{equation}
    \text{d}s^{2} = a(\eta)^{2}\left(-\text{d}\eta^{2}+frac{\text{d}r^{2}}{1-kr^{2}}+r^{2}\text{d}\Omega_{(2)}^{2}\right)
    \label{eq:FRW_conf}
\end{equation}
the quantity within parentheses is independent of the conformal time. We could try with a hand-waving approach and obtain (out of thin air) the expression for the cosmological redshift, but we'll take a somewhat more rigorous approach.
\par Consider the four-velocity of comoving observers $U^{\mu} = (1,0,0,0)$, then the metric admits a Killing tensor of the form 
\begin{equation*}
    K_{\mu\nu} = a^{2}(g_{\mu\nu}+U_{\mu}U_{\nu})
\end{equation*}
that you can check that it satisfies $D_{(\sigma}K_{\mu\nu)} = 0$ where $D_{\sigma}$ is the covariant derivative. By definition of Killing vector (the generalization to a K. tensor is straightforward), the quantity 
\begin{equation}
    K^{2} = K_{\mu\nu}V^{\mu}V^{\nu} = a^{2}\left[V_{\mu}V^{\mu}+(U_{\mu}V^{\mu})^{2}\right]
    \label{eq:conserved_killing}
\end{equation}
is conserved along the geodesics. Here $V^{\mu} = \text{d}x^{\mu}/\text{d}\lambda$. For massive particles, since $V_{\mu}V^{\mu} = -1$ (note the convention on the signature of the metric), we also have
\begin{equation*}
    (V^{0})^{2} = 1+|\vec{V}|^{2} \quad |\vec{V}|^{2} = g_{ij}V^{i}V^{j}
\end{equation*}
Recalling the for of the four-velocity of the observer, we have $U_{\mu}V^{\mu} = -V^{0}$, so (\ref{eq:conserved_killing}) implies
\begin{equation*}
    |\vec{V}| = \frac{K}{a}
\end{equation*}
Notice how this means that the particle "slows down" with respect to the comoving coordinates as the Universe expands. If we instead consider null geodesics (this is what we're actually interested in)
\begin{equation*}
    V_{\mu}V^{\mu} = 0 \implies U_{\mu}V^{\mu} = \frac{K}{a}
\end{equation*}
but for null-like particles, like photons, we can write the four-velocity as $V^{\mu} = (\omega, \vec{k})$. So the frequency of the photon as measured by a comoving observer is $\omega = -U_{\mu}V^{\mu}$. The frequency emitted will still be given by the same functional expression but with in the local inertial frame of the emitter, so $\omega_{em} = -K/a_{em}$. We then find out that 
\begin{equation*}
    \frac{\omega_{obs}}{\omega_{em}} = \frac{a_{em}}{a_{obs}}
\end{equation*}
To make things easier we then define a quantity called \emph{redshift} defined as 
\begin{equation*}
    z_{em} = \frac{\lambda_{obs}-\lambda_{em}}{\lambda_{em}}
\end{equation*}
so that, if the observation takes place today ($a_{obs} = a_{0} = 1$), we can just write 
\begin{equation*}
    a_{em} = \frac{1}{1+z_{em}}
\end{equation*}
and 
\begin{equation}
    \boxed{
    \frac{\nu_{obs}}{\nu_{em}} = \frac{1}{1+z_{em}}
    }
    \label{eq:cosmological_redshift}
\end{equation}
So the redshift tells us two things: The scale factor of the universe when the photon was emitted and, in a sense, the distance of the emitter when the photon was emitted. In fact, using Hubble's law (assuming we've somehow computed the Hubble constant at its present value) we can easily calculate the \emph{instantaneous physical distance}
\begin{equation*}
    v = cz = H_{0}d_{P}
\end{equation*}
Was all of this really necessary for what we'll be going through in the next sections? No, not really. But I thought it might have been cool having a clue on where this curious quantity called redshift comes from and, more importantly, knowing what it actually means and implies. That being said, we can hop on back on tracks.
\subsection{Keplerian Motion}
Let us consider two point-like objects of mass $M_{1}$, $M_{2}$ moving along elliptical orbits around their center of mass as shown in Fig.\ref{fig:keplerianmotion}.
\par As customary, we can define the total mass $M_{tot}$ and the reduced mass $\mu$ of the system as weoo as the relative coordinate $\vec{r} = \vec{r}_{1}-\vec{r}_{2}$, with $\vec{r}_{i}$ the distance of the mass $M_{i}$ from the center of mass.
\begin{figure}[h!]
    \centering 
    \includegraphics[width=0.8\textwidth]{img/keplerianmotion.png}
    \caption{Binary system religiously moving as predicted by Newtonian dynamics.\\
    Credits: \cite{celoria2018lecturenotesblackhole}.}
    \label{fig:keplerianmotion}
\end{figure}
\begin{equation*}
    \vec{r}_{1} = \frac{M_{2}}{M_{tot}} \vec{r} \quad \vec{r}_{2} = -\frac{M_{1}}{M_{tot}} \vec{r} 
\end{equation*}
The velocities relative to the center of mass can be obtained straightforwardly by deriving the expression above in respect to time.
\par The total energy of the system is 
\begin{equation*}
    E = \frac{M_{1}v_{1}^{2}}{2}+\frac{M_{2}v_{2}^{2}}{2} -\frac{GM_{1}M_{2}}{r} = \frac{\mu v^{2}}{2}-\frac{GM\mu}{r} = -\frac{GM\mu}{a}
\end{equation*}
In the last step, we have used the conservation of energy along the orbit and wrote the energy in terms of the semi-major axis $a$. Hence, the binary system is equivalent to considering a single body with mass $\mu$ moving in an effective external gravitational potential. The orbital frequency is given by Kepler's third law 
\begin{equation*}
    \omega = \frac{2\pi}{T} = \sqrt{\frac{GM}{a^{3}}}
\end{equation*}
which is, incidentally, the same result you obtain considering the proper-orbital period of a test-body in the gravitational potential induced by a Schwartzschild metric\footnote{It's not difficult to show. You have to make use of the various conserved quantities the Killing vectors kindly gift you and then calculate $\text{d}\phi/\text{d}\tau$ with $\tau$ the proper time of the particle.}.
\par It is useful to calculate the orbital angular momentum $J_{orb} = \vec{r}\wedge\mu\vec{v}$, which has magnitude
\begin{equation*}
    J_{orb} = \mu\sqrt{GMa(1-e^{2})}
\end{equation*}
where $e$ is the eccentricity of the orbit. Thus, for a circular orbit ($e = 0$) the magnitude of the angular momentum is just 
\begin{equation*}
    J_{orb, C} = \mu\sqrt{GMr_{C}}
\end{equation*}
hence the tangential velocity is
\begin{equation*}
    v = \sqrt{\frac{GM}{r_{C}}}
\end{equation*}
\subsection{GW radiation from a binary system}
As you should probably know if you've attended a GR class, a system of sufficiently\footnote{Please note that there's not a threshold, proper, on the mass or density of the objects to start emitting GW, but rather their intensity would be so faint that we couldn't possibly dream about observing them.} compact objects bound in a binary system can start emitting GWs. A full description of how GWs are produced in the weak field limit from the linearized Einstein's equation is beyond the scope of these notes. The interested reader can refer to \cite{carroll}, §7.
\par For Newtonian sources localized in a compact region of space, the gravitational power radiated is governed by the \emph{quadrupole momenta} of the binary and given by 
\begin{equation}
    \frac{\text{d}E_{rad}}{\text{d}t} = \frac{G}{5}\langle\dddot{I_{ij}}\dddot{I_{ji}}\rangle
    \label{eq:quadrupole_rad}
\end{equation}
where the brackets stand for average over the solid angle and the tensor $I_{i}$ is the mass quadrupole moment given by the following integral
\begin{equation*}
    I_{ij}(t) = \int_{\text{source}} \text{d}^{3}r\,\left(r_{i}r_{j}-\frac{1}{3}\delta_{ij}r^{2}\right)T_{00}
\end{equation*}
\par For binary systems, it is easy to prove that the averaged power emitted can be expressed as 
\begin{equation*}
     \frac{\text{d}E_{rad}}{\text{d}t} = \frac{32}{5}\frac{G^{4}}{c^{5}}\frac{\mu^{2}M^{3}}{a^{5}}F(e)\
\end{equation*}
where the factor 
\begin{equation*}
    F(e) = (1-e^{2})^{-7/2}\left(1+\frac{73}{24}e^{2}+\frac{37}{96}e^{4}\right)
\end{equation*}
depends on eccentricity only and shows that highly eccentric binaries are much more efficient at radiating away energy in form of GWs.
\par We can also compute the averaged angular momentum flux 
\begin{equation*}
    \frac{\text{d}J_{rad}}{\text{d}t} = \frac{32}{5}\frac{G^{7/2}}{c^{5}}\frac{M_{1}^{2}M_{2}^{2}(M_{1}+M_{2})^{1/2}}{a^{7/2}}(1-e^{2})^{-2}\left(1+\frac{7}{8}e^{2}\right)
\end{equation*}
So GWs extract both energy and angular momentum out of the binary. As a consequence, the two masses are drawn closer, spiraling around each other until they eventually merge.
\par From the previous equations is possible to show that 
\begin{align}
    \frac{\text{d}a}{\text{d}t} &= -\frac{64}{5}\frac{G^{3}}{c^{5}}\frac{M_{1}M_{2}(M_{1}+M_{2})}{a^{3}}F(e)\\
    \frac{\text{d}e}{\text{d}t} &= -\frac{304}{15}\frac{G^{3}}{c^{5}}\frac{M_{1}M_{2}(M_{1}+M_{2})}{a^{4}}e(a-e^{2})^{-5/2}\left(1+\frac{121}{304}e^{2}\right)
    \label{eq:orb_variation}
\end{align}
from which we see that GWs drive the binary towards circularization along the inspiral.
\par We can integrate in time the expression for the separation and, neglecting that the eccentricity varies in time, find an estimate for the \emph{coalescence time} of the binary 
\begin{equation}
    t_{\text{coal}} = \frac{5}{256}\frac{c^{5}}{G^{3}}\frac{a_{0}^{4}}{M_{1}M_{2}(M_{1}+M_{2})}\frac{1}{F(e)}
    \label{eq:coalescence_time}
\end{equation}
We can then rewrite the initial separation $a_{0}$ as 
\begin{equation*}
    a_{0} = 1.6R_{\odot}\left(\frac{M_{1}}{M_{\odot}}\right)^{3/4}\left[q(1+q)F(e)\left(\frac{t_{\text{coal}}}{1\,\text{Gyr}}\right)\right]^{1/4}
\end{equation*}
where $q = M_{2}/M_{1}$. From the expression above it's immediate to see that the initial separation required for two objects bound in a binary to merge within an Hubble time\footnote{The Hubble time, if you recall, is $t_{H} = H_{0}^{-1}\approx 14.4\,\text{Gyr}$ and is a decent approximation of the age of the Universe.} is $a_{0}\approx 10^{11}\,\text{cm}\approx 0.01\,\text{AU}$, which is quite the narrow range, actually.
\par This is also quite problematic if you think about it. For once, during their evolution, stars undergo a giant phase characterized by $R_{G} \approx 10^{14}\,\text{cm}$, so stars with initial separation $a_{0}<R_{G}$ would simply engulf each other during their giant phase and merge in a single star \emph{before} forming our nice binary of compact objects.
\par The scope of the next sections is trying to find a way to have our compact objects come sufficiently close so that they can get bound in a binary system and then, eventually, merge.
\section{Common Evolution Channel (SBHs)}
As you may have guessed from the title of this section, we're now turning our attention to Stellar-mass BHs; in the following, we'll try to elaborate and write down a sensible mechanism to bring the two objects close enough to form our cute binary. Two formation channels consistent with
the first GWs observations of SBHB mergers have been proposed: the \emph{common evolution of field binaries} and the \emph{dynamical capture in dense
environments}. Let's focus on the first one.
\par The common evolution is the astrophysical scenario in which the two stars, eventually producing the SBHB, form as a stellar binary system and evolve together through the different phases of stellar evolution. To achieve this, we have to look into four key ingredients, which will be the subject of the four next subsections.
\subsection{Gravitational potential}
As we've already done somewhere above, let us consider two point-like objects of mass $M_{1}$ and $M_{2}$ separated by $a$ and moving, for simplicity, in circular orbits about their common center of mass. In the corotating frame, the energy potential of the system is just 
\begin{align*}
    U &= -\frac{Gm(M_{1}+M_{2})}{r_{CM}}-\frac{1}{2}m\omega^{2}r_{CM}^{2}\\
    &= -Gm\left(\frac{M_{1}}{s_{1}}+\frac{M_{2}}{s_{2}}\right)-\frac{1}{2}m\omega^{2}r_{CM}^{2}
\end{align*}
where $s_{i}$ is the distance of a test-mass $m$ from mass $M_{i}$ and $r_{CM}$ is the distance of said test-mass from the center of mass.
\par As we've already see in the last chapter, the equilibrium points are the Lagrange points of the potential, which satisfy 
\begin{equation*}
    \vec{F} = -m\nabla\,\phi = 0
\end{equation*}
There exists a critical equipotential surface, forming a two-lobed figure-of-eight, with one of the two object at the centre of each lobe and intersecting itself at the $L_{1}$ Lagrangian point, known as Roches lobes. An approximate formula for the radius of the Roches lobe around $M_{1}$ was derived by Eggleton 
\begin{equation*}
    \frac{R_{1}}{a} = \frac{0.49q^{2/3}}{0.6q^{2/3}+\ln(1+q^{1/3})}
\end{equation*}
up to a 1\% accuracy.
\subsection{Mass transfer}
As we've already seen, when one of the two objects fills its Roche lobe, matter may overflow the lobe and infall onto the other object, without any need of energy exchange.
\par Suppose that $M_{2}$ loses material at a rate $\dot{M}_{2}<0$ and let $\beta \in [0,1]$ be the fraction of the ejected matter leaving the system, so that 
\begin{equation*}
    \dot{M}_{1} = -(1-\beta)\dot{M}_{2} > \geq 0
\end{equation*}
Clearly, if $\beta = 0$, then all the mass lost by $M_{2}$ is captured by $M_{1}$ and the mass transfer is fully conservative. Keeping it a free parameter, we are instead considering the more
general case in which a fraction of the mass can be lost and escape the system, as may be the case, for example, for stellar winds.
\par The angular momentum was $J = \mu\sqrt{GMa}$; if we now differentiate it in respect to time and use $\dot{M}_{1}+\dot{M}_{2} = \dot{M} = \beta\dot{M}_{2}$, we obtain 
\begin{equation}
    \frac{\dot{a}}{a} = -2\left(1-\frac{M_{2}}{M_{1}}\right)\frac{\dot{M}_{2}}{\dot{M}_{1}}
\end{equation}
In short, if $M_{1}>M_{2}$, the obrit expands ($\dot{a}>0$), otherwise it shrinks. From Kepler's third law we also know that 
\begin{equation*}
    \frac{\dot{\omega}}{\omega} = -\frac{3}{2}\frac{\dot{a}}{a}
\end{equation*}
and thus the angular frequency unsurprisingly increases as the orbit shrinks.
\subsection{Supernova kicks}
At the end of all the subsequent stages of nuclear burning, there's a chance that the more massive stars can undergo a supernova (SN) explosion\footnote{As shown in Chapter \ref{ch:v}, §6.3.2, SN explosions are described fairly well with the Sedov-Taylor blastwave model.}, after which much of the stellar material is expelled and leaving behind only a shell of its former self, typically in the form of a neutron star or a stellar-mass black hole. The mass loss is practically instantaneous as the typical timescale for the explosion is much shorter than the orbital period.
\par In general, the collapse is not perfectly
symmetric nor isotropic. As a result, the SN imprints a kick to the object characterised by a recoil velocity $v_{\text{kick}}$. Since mass ejection decreases the total mass of the binary, also the gravitational potential changes and, if enough mass is ejected, the SN explosion can unbind the binary (which would suck for our purposes' sake).
\par Moreover, $v_{\text{kick}}$ is generally (much) greater than the orbital velocity, so the kick may as well destroy most of the binaries.
\par To describe the effect of a SN explosion, we start by considering the setup we've used before, the notation is identical.
\par Prior to the SN-explosion, the relative velocity is 
\begin{equation*}
    v_{i} = \sqrt{\frac{G(M_{1}+M_{2})}{a_{i}}}
\end{equation*}
After the explosion of, say, the giant star $M_{1}$, what remains of that star has mass $M_{c}<M_{1}$, and $\Delta M = M_{1}-M_{c}$ is ejected. We can assume that, after an instantaneous explosion, the position of the once-$M_{1}$ has not changed, but the reduced mass has
\begin{equation*}
    \mu_{f} = \frac{M_{c}M_{2}}{M_{c}+M_{2}}
\end{equation*}
and the final (relative) velocity is $\vec{v}_{f} = \vec{v}_{i}+\vec{v}_{\text{kick}}$. The final energy of the sytem will then just be
\begin{equation*}
    E_{f} = \frac{1}{2}\mu_{f}v_{f}^{2}-\frac{GM_{c}M_{2}}{a_{i}}
\end{equation*}
The system will remain bound if the final velocity is smaller than the escape velocity
\begin{equation*}
    v_{f}\leq v_{e} = \sqrt{\frac{2G(M_{c}+M_{2})}{a_{i}}}
\end{equation*}
If the SN-explosion is perfectly spherically symmetric, isotropy leads to no kick at all, and the condition for "boundness" implies 
\begin{equation*}
    \Delta M \leq \frac{M_{1}+M_{2}}{2}
\end{equation*}
Even in absence of kicks, a binary can be disrupted due to mass loss only, if the SN explosion ejects more than half of the initial mass of the binary
system.
\par The case of asymmetric SNs is more complicated and usually characterized by kicks of order $10^{2}\,\text{km}/\text{s}$. More precisely, the natal
kick distribution is typically modelled by a Maxwellian probability distribution with velocity dispersion $\sigma_{v} = 190\,\text{km}/\text{s}$, even though, a bimodal distribution may be somewhat more appropriate, especially when describing neutron stars' kicks. For SBHs the situation is less clear and the problem of binary disruption due to SN kicks might be less severe. In this case, there is no hard surface to bounce onto and, as a consequence, the kicks might be smaller, typically of order of $50\,\text{km}/\text{s}$, which is, however, of the order of the the escape velocity for the most of the globular clusters in the Milky Way.
\section{Common Envelope}
Coming soon.