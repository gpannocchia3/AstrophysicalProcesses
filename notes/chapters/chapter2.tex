
%============================= HEADER =================================
\chapter{The Einstein coefficients}\label{ch:2}

\section{Introduction}
Kirchhoff's law (eq.\ref{Kirchhoff}), which relates the (spontaneous) emission and the absorption coefficient, seems to imply some underlying microscopic connection between the two phenomena.
\par As was first discovered by Einstein, that is exactly the case. Let's consider a two level atom interacting with radiation.
\begin{figure}[h!]
    \centering 
    \includegraphics[width=0.9\textwidth]{img/twolevels.png}
    \caption{Photon emission and absorption in a two levels atom. \\ Credits: G. Rybicki, A. Lightman.}
    \label{fig:twolevels}
\end{figure}
As depicted in Fig.\ref{fig:twolevels}, we'll consider two discrete energy levels: the lower with energy $E$ and degeneracy $g_{1}$, while the upper level has energy equal to $E+h\nu_{0}$ and degeneracy $g_{2}$. Transition between the two levels is possible only through absorption ($1\to 2$) or emission ($2\to1$) of photons of energy $h\nu_{0}$.
\par Three processes can be thus identified: \emph{spontaneous emission}, \emph{stimulated emission} and \emph{absorption}.
\subsection*{Spontaneous Emission}
The process we'll refer to as \emph{spontaneous emission} occurs when the system transitions from the excited state 2 to the lower level 1 through the emission of a photon. 
\par Spontaneous emission can occur even in the \textbf{absence of radiation fields} and can be assumed to be \textbf{isotropic}. The transition probability for spontaneous emission is defined through the \emph{Einstein A-coefficient} $A_{21}$, which has units $[A_{21}] = \text{s}^{-1}$.
\subsection*{Absorption}
In the presence of a radiation field with the right energy, the system can absorb a photon to transition from state 1 to a higher energy state. Assuming that the radiation field cannot self-interact, we expect the probability per unit time to be proportional to the density of photons (or to the mean intensity $J_{\nu}$) at frequency $\nu_{0}$.
\par However, the energy difference between the two states is not infinitely sharp, but more like a smoother curve we'll call the \emph{line profile function $\phi(\nu)$}, 
\begin{figure}[h!]
    \centering 
    \includegraphics[width=0.9\textwidth]{img/lineprof}
    \caption{Line profile for a two levels atom. \\ Credits: G. Rybicki, A. Lightman.}
    \label{fig:lineprof}
\end{figure}\\
which is conveniently peaked at $\nu = \nu_{0}$ and is correctly normalized
\begin{equation*}
    \int_{0}^{+\infty} \phi(\nu)\,\text{d}\nu = 1
\end{equation*}
Not bothering for the moment with the underlying physical mechanisms that concur in determining the line profile, we'll propose the following definition for the transition probability for absorption: $B_{12}\overline{J}$, where $\overline{J}$ is
\begin{equation*}
    \overline{J} = \int_{0}^{+\infty} J_{\nu}\phi(\nu)\,\text{d}\nu
\end{equation*}
and $B_{12}$ is the \emph{Einstein B-coefficient}.
\subsection*{Stimulated Emission}
As anticipated, there exists yet another way for a system to emit a photon, but this time requiring the presence of a radiation field. Taken $\overline{J}$ has the same meaning as before, we'll define the transition probability per unit time for stimulated emission as $B_{21}\overline{J}$.
\par If we were to assume that the mean intensity $J_{\nu}$ changes slowly over the width $\Delta\nu$ of the line profile, we could in principle approximate the line profile as a $\delta$-function peaked at $\nu_{0}$. \par Some books use the energy density $U_{\nu}$ in the definitions. It's pretty much the same since they carry the same information, but be aware that the two definitions will differ of a $c/4\pi$ factor.
\section{Relations between the Einstein Coefficients}
It would be most useful to know if there were some kind of relations between the three different coefficients, and if there were some way of relating them to what we've called $\alpha_{\nu}$ and $j_{\nu}$ when dealing the radiative transport. 
\par If this weren't the case, this section would have no reason to exist, so it's safe for you to assume that such relations do in fact exist. To do so, however, we have to invoke the \emph{quantum theorem of detailed balance} (see, for example, \cite{goodstein}), which roughly says 
\begin{quotation}
    If states $a$ and $b$ of a system have the same energy, then if $P_{ab}$ is the probability per unit time of a transition from $a$ to $b$, and $P_{ba}$ from $b$ to $a$,
    $$ P_{ab} = P_{ba} $$
\end{quotation}
For an atom of matter to be in equilibrium with a radiation field, then, the probability of said atom being in the ground state and absorbing a photon of a given frequency must be equal to the probability that it is in the excited state and emits the photon.
\par If we assume to be in a steady-state condition, the rate of transition from level 1 to level 2 has to be equal to the rate of transition from level 2 to level 1, or, more generally
\begin{equation*}
    n_{i}\sum_{j}R_{ij} - \sum_{j}n_{j}R_{ji} = 0
\end{equation*}
If we put in the transition rates we've written down earlier, we get 
\begin{equation*}
    n_{1}B_{12}\overline{J} = n_{2}A_{21}+n_{2}B_{21}\overline{J}
\end{equation*}
which means that the number of transitions per unit time per unit volume out of state 1 must be equal to the number of transitions per unit time per unit volume into state 1.
\par Solving for $\overline{J}$
\begin{equation}
    \overline{J} = \frac{n_{2}A_{21}}{n_{1}B_{12}-n_{2}B_{21}} = \frac{A_{21}/B_{21}}{(n_{1}/n_{2})\cdot(B_{12}/B_{21})-1}
\end{equation}
In thermodynamic equilibrium we can use eq.\ref{boltmann-occ} to relate the occupation numbers
\begin{equation*}
    \frac{n_{1}}{n_{2}} = \frac{g_{1}}{g_{2}}\frac{\exp\left(-E/kT\right)}{\exp\left[-(E+h\nu_{0})/kT\right]} = \frac{g_{1}}{g_{2}}\exp\left(h\nu_{0}/kT\right)
\end{equation*}
but in thermodynamic equilibrium we know that $J_{\nu} = B_{\nu}$, so if $B_{\nu}$ varies slowly on the scale of $\Delta\nu$ we can assume $\overline{J}\approx B_{\nu}(\nu_{0})$.
\par That means the following relations must simultaneously hold
\begin{align}
    g_{1}B_{12} &= g_{2}B_{21} \\
    A_{21} &= \frac{2h\nu^{3}}{c^{2}}B_{21}
    \label{eq:einsteinrelations}
\end{align}
which are known as \emph{Einstein relations}. They connect atomic properties $A_{21}$, $B_{21}$ and $B_{12}$ and have no reference to the temperature $T$ (unlike Kirchhoff's law).
\par Although we have derived eq.\ref{eq:einsteinrelations} assuming thermodynamic equilibrium, those two relations \textbf{must be always valid} whether or not the atoms are in thermodynamic equilibrium.
\subsection{Absorption and Emission coefficients in terms of Einstein coefficients}
To obtain the emission coefficient $j_{\nu}$ we have to make a crucial assumption about the frequency distribution of the emitted radiation: This emission is distributed with the same line profile $\phi(\nu)$ that describes absorption, which is often verified in astrophysics (good for us).
\par By the definition (\ref{eq:jnu}), we already know the amount of energy emitted in volume $\text{d}V$, solid angle $\text{d}\Omega$, frequency $\text{d}\nu$, and time $\text{d}t$. Now, since each atom contributes with $h\nu_{0}$ to spontaneous emission over a $4\pi$ solid angle, we can express the emission coefficient as 
\begin{equation}
    j_{\nu} = \frac{h\nu_{0}}{4\pi}n_{2}A_{21}\phi(\nu) 
    \label{eq:jnueinstein}
\end{equation}
Similarly, you can show that the energy absorbed from radiation in frequency range $\text{d}\nu$, solid angle $\text{d}\Omega$, time $\text{d}t$ and volume $\text{d}V$ is
\begin{equation*}
    \frac{h\nu_{0}}{4\pi}n_{1}B_{12}I_{\nu}\phi(\nu)\,\text{d}\nu\,\text{d}\Omega\,\text{d}t\,\text{d}V
\end{equation*}
From here follows immediately the \emph{true absorption coefficient}
\begin{equation}
    \alpha_{\nu} = \frac{h\nu_{0}}{4\pi}n_{1}B_{12}\phi(\nu) 
\end{equation}
Note that since stimulated emission is proportional to the specific intensity $I_{\nu}$ and only affects the photons along the given beam, it's not a big stretch to claim that it behaves in a way that is akin to true absorption. We shall then define the \emph{absorption coefficient, corrected for stimulated emission} as
\begin{equation}
    \alpha_{\nu} = \frac{h\nu_{0}}{4\pi}\phi(\nu)(n_{1}B_{12}-n_{2}B_{21})
    \label{eq:alphacorrected}
\end{equation}
in which we're regarding stimulated emission like a \emph{negative absorption}.
\par In terms of the newly defined emission and absorption coefficients, the equation of radiative transfer takes a new look 
\begin{equation}
    \frac{\text{d}I_{\nu}}{\text{d}s} = -\frac{h\nu_{0}}{4\pi}\phi(\nu)(n_{1}B_{12}-n_{2}B_{21})+\frac{h\nu_{0}}{4\pi}n_{2}A_{21}\phi(\nu) 
\end{equation}
\par Computing the ratio $j_{\nu}/\alpha_{\nu}$ with the new definitions holds
\begin{equation}
    S_{\nu} = \frac{n_{2}A_{21}}{n_{1}B_{12}-n_{2}B_{21}} = \frac{2h\nu^{3}}{c^{2}}\left(\frac{g_{2}n_{1}}{g_{1}n_{2}}-1\right)^{-1}
    \label{eq:generalkirch}
\end{equation}
where the last equality is obtained just by substitution of the Einstein relations (\ref{eq:einsteinrelations}). Note that eq.(\ref{eq:generalkirch}) is a generalized Kirchhoff's law. Let's now consider three interesting cases sprouting from this last equality.
\subsubsection*{Thermal Emission (LTE)}
If the matter is in local thermal equilibrium (LTE) with itself (but not necessarily with the radiation), eq.\ref{boltmann-occ} must hold locally.
In this case, we correctly retrieve $$ S_{\nu} = B_{\nu}$$
\subsubsection*{Nonthermal Emission}
We can't assume eq.\ref{boltmann-occ} to hold, and particles do not obey the Maxwellian distribution or any of the fancy properties we'd like them to.
\subsubsection*{Inverted Populations: MASERS}
Let's consider the definition of the absorption coefficient corrected for stimulated emission (\ref{eq:alphacorrected}). Whether the coefficient is positive or negative depends on the term within parentheses, which can be restated using Einstein's relations 
\begin{equation*}
    \frac{n_{1}g_{2}}{n_{2}g_{1}}-1 > 0
\end{equation*}
so that when this relation is satisfied, even out of thermal equilibrium, we say that the system has \emph{normal populations}. However, it is possible to put enough atoms in the upper state so that we achieve what is called an \emph{inverted population}
$$ \frac{n_{1}}{g_{1}} < \frac{n_{2}}{g_{2}}$$
which is quite the odd configuration, if you think about it. We're basically saying that a higher energy level is more densely populated than a lower energy one, which, by means of eq.\ref{boltmann-occ}, corresponds to stating that the temperature $T$ is negative, since $E$ is assumed to be non-negative.
\par And no, $T$ is not in Celsius degrees.
\par In such a scenario, the absorption coefficient is negative; rather than being damped to extintion, the intensity of the radiation ray \emph{increases} when passing through matter. Such a system is called MASER (Microwave Amplification by Stimulated Emission of Radiation). For visible light you'd get what we usually call "laser".
\par The funny thing is that we actually observe this phenomenon occurring spontaneously in Nature, like for example in some molecular clouds formed in the ISM. Some sources in specific molecular lines (like the OH lines) were found to be \emph{abnormally high}. Let me quantify how much "abnormal" we're talking about.
\par Your typical molecular cloud has a density a bit shy of $10^{9}$ particles per $\text{m}^{3}$ and temperature in the range of $10-30\,\text{K}$, so kinda chilly.
\par If those "abnormal" sources were assumed to be optically thick in the spectral lines and the specific intensity was equated to $B_{\nu}$, you'd probably roll down you chair reading on your computer that those sources should have temperatures as high as $10^{9}\,\text{K}$.
\par Not so chilly anymore, huh?
\par The favored explanation for this weird phenomenon (and most likely the correct one) makes use exacly of what we've just shown: Maser action.
\par If you want to read something (slightly) more rigorous, you can look at \cite{choudhuri}, §6.6.3.
\par For those among you that, like me, were particularly invested in TV shows like \emph{X-Files} and similar, may be interested to know that one of the most likely explanations for the so-called \href{https://en.wikipedia.org/wiki/Wow!_signal}{Wow! Signal}\footnote{This was brought to my attention by a fellow colleague of the Aerospace Engineering department, and I thought it was well-worth sharing it with you. Thanks, Monta.} is in fact attributed to MASER action on a Hydrogen cloud.
\section{Hydrogenoid Atoms}
Fairly interesting objects to study are \emph{hydrogenoid atoms}, that is atoms with $Z$ protons but that had ended up (no matter how) with only one $e^{-}$. For this type of atoms, Quantum Mechanics predicts energy levels distributed as such 
\begin{equation*}
    E_{n} = -\frac{Z^{2}\,\text{Ry}}{n^{2}} \quad \text{Ry} = \frac{m_{e}e^{4}}{2\hbar^{2}} \approx 13.6\,\text{eV}
\end{equation*}
\par If we were to introduce relativistic corrections through perturbation theory, the expression would get a bit more complicated
\begin{equation*}
    E(n,\,j) = -\frac{Z^{2}\,\text{Ry}}{n^{2}}\left(1+\alpha^{2}\left[\frac{n}{j+1/2}-3/4\right]\right)
\end{equation*}
introducing \emph{fine splitting} of the energy levels, that now depend on the value of the angular momentum. It is probably worth pointing out that the energy levels are still degenerate in the magnetic quantum number $m$. Given a problem with spherical symmetry, it would be beneath us to claim that a scalar perturbation could be able to completely solve the degeneracy\footnote{The system is still invariant under rotations.}. Degeneracy is solved completely only if we introduce in the system a preferred direction\footnote{cfr. \emph{Stark-Lo Surdo Effect} and the \emph{Zeeman effect}.}.
\par Depending on the value of the orbital angular momentum eigenvalue $l$ we can assign a name to distinguish them:
\begin{itemize}
    \centering
    \item $l = 0 \to$ $s$-orbitals
    \item $l = 1 \to$ $p$-orbitals
    \item $l = 2 \to$ $d$-orbitals
    \item $l = 3 \to$ $f$-orbitals
\end{itemize}
This way, transitions may be represented in a \emph{Grotrian diagram} (Fig.\ref{fig:grotrian}), where the energy of a given level is plotted in function of the $l$ eigenvalue, through which selection rules are made most evident.
\par What happens if we are to consider atoms with more than one electron? Will our model still work? Nope, not a chance.
\par On top of getting stupidly difficult to evaluate even the non relativistic limit, there are \emph{many more ways} to couple interactions, resulting in more and more sources of opacity. It is actually elements with $Z>1$ and $\#e^{-} > 1$ that make the greatest contributions to opacity, especially \emph{metals} ($Z\geqslant 3$).
\begin{figure}[h!]
    \centering
    \includegraphics[width=0.7\textwidth]{img/Grotrian_H.png}
    \caption{Example of Grotrian Diagram for Hydrogen. Credits: Wikipedia.}
    \label{fig:grotrian}
\end{figure}
\par Sometimes it is useful to express the electron configuration by means of the \emph{Russel-Saunders notation}. Said $S$, $L$, $J$ respectively the spin, the orbital angular momentum and the total angular momentum eigenvalues, we can express a given electron configuration in such a fashion
\begin{equation*}
    \,^{2S+1}L_{J}
\end{equation*}
where $L$ is usually written in spectroscopic notation, that is the symbol associated to the orbital with that $L$ value.
\subsection{Hyperfine Transition}
What we'll be most interested in considering, however, is the hyperfine splitting that arises if we consider the coupling between the spin of the nucleus and the electron.
\par If we were to consider such interaction, we'd find a hyperfine splitting of the lowest energy state. We'll be interested in the energy of the radiation coming from an electron spin flip: The system will transition from a state with total spin $F = 1$ to a one with $F=0$, resulting in the emission of radiation (Fig.\ref{fig:spinflip}).
Ignoring higher order corrections to the Hamiltonian, the radiation resulting from the spin flip will have a wavelength 
\begin{equation}
    \lambda_{hf} = \frac{3\pi\hbar^{5}m_{p}c^{3}}{g_{e}g_{p}m_{e}^{2}e^{8}}\approx 21.106114054160(30)\,\text{cm}
    \label{eq:21cm}
\end{equation}
where the expression is in CGS units and $g_{e}$, $g_{p}$ are respectively the electron and the proton spin g-factors ($\approx 2$ for the electron, $\approx 5.59$ for the proton).
\par Unfortunately, such a transition cannot be observed in lab-experiments. The extimated half-life time of the transition from the Einstein $A$ coefficient\footnote{Remember that $A$ has the dimensions of the inverse of a time, so $\tau_{HL}\approx A^{-1}$.} is of the order of a few Myrs. Which means that unless you have \emph{a lot} of time to spare to fight against death, you'll probably never record even a single one of these transitions. And that is ignoring the possibility of Hydrogen getting collisionally de-excited, which is unfathomably more probable.
\begin{figure}[h!]
    \centering
    \includegraphics[width=0.4\textwidth]{img/hlineflip.png}
    \caption{Schematic representation of the spin flip and the resulting emission.\\ Credits: Wikipedia.}
    \label{fig:spinflip}
\end{figure}
\par This, however, has a strikingly beautiful consequence: Although "rare", we should be able to detect the 21-cm emission line if we were to look somewhere where the transition may have happened millions of years ago.
\par If you were to steal a radiotelescope\footnote{Do not recommend. They get awfully big if you want to see something cool.} and point it towards the Galactic plane, you'll see a really sharp line in proximity to the 21-cm line ($\sim 1420\,\text{MHz}$), as shown in Fig.\ref{fig:21cm}.
\begin{figure}[h!]
    \centering
    \includegraphics[width=0.8\textwidth]{img/hydrogenspec.png}
    \caption{Sampling of the 21-cm emission line with the new Spider500 Radio telescope of the Physics Department of the University of Pisa.\\ Note how the sharp, black emission line is not centered on 1420 MHz due to Doppler shift (see next section).\\ Credits: AMLab, Group 8, a.y. 2025/2026.}
    \label{fig:21cm}
\end{figure}
\section{Line Broadening}
It would be most dumb for us to assume that energy levels, or the lines connecting them, are infinitely sharp. Even a quick look at Heisenberg's uncertainty principle should convince you otherwise.
\par When we defined the Einstein's coefficients, we had to introduce the line profile $\phi(\nu)$ to account for the non-zero width of the line, and it's about time we consider some of the phenomena that concur in determining the line's actual shape.
\subsection{Natural Broadening}
For this one, you'll have to recall some of the basics results of Quantum Mechanics. The calculation is not particularly difficult, but it's surely long, so I'm going to omit that and take a simpler route.
\par Note that the spontaneous decay of an atomic state $n$ proceeds at a rate
\begin{equation*}
    \Gamma = \sum_{m} A_{nm}
\end{equation*}
where the sum is over all states $m$ of lower energy (which may be quite a few). If radiation is present, we should add the induced rates to this. The coefficient of the wave function of state $n$, therefore, is of the form $e^{-\Gamma t/2}$ and leads to a decay of the electric field by the same factor. We have an emitted spectrum determined by the decaying sinusoid type of electric field, so the line profile must be a \emph{Lorentz or Natural profile}\footnote{Fun fact: The Lorentzian distribution (also known as Cauchy distribution among mathematicians) is a classical example of a pathological distribution. In fact the Cauchy distribution  has no mean, variance or higher momenta defined. Its mode and median are well defined and are both equal to $\nu_{0}$.}
\begin{equation*}
    \phi(\nu) = \frac{\Gamma}{4\pi^{2}}\frac{1}{(\nu-\nu_{0})^{2}+\left(\frac{\Gamma}{4\pi}\right)^{2}}
\end{equation*}
More often than not, however, we're going to assume that the effect of natural broadening is rather peaked around $\nu_{0}$, similar to a $\delta$-function, since, at least as far as astrophysics is concerned, there are more relevant sources of broadening.
\subsection{Doppler Broadening}
Consider an atom in thermal motion, so that by simple relativistic considerations the frequency of emission or absorption in its own frame corresponds to a different frequency for the observer. Each atom will have its own Doppler shift, so that the net effect is to spread the line out, but not to change its total strength.
\par Recall the classic Doppler effect 
\begin{equation}
    \nu -\nu_{0} = \nu_{0}\frac{v_{z}}{c}
\end{equation}
Here $\nu_{0}$ is the rest frequency. Note that this can be extended to \emph{bulk motion} as well. We can decompose the velocity as
\begin{equation*}
    \vec{v} = \langle\vec{v}\,\rangle + \delta\vec{v}
\end{equation*}
which goes by the name of \emph{Reynolds' decomposition} in Fluid-dynamics books. Generally, an ensemble of atoms will have different $\delta\vec{v}$, so each atom will absorb photons at different frequencies.
\par What we have to do is then convolute the Doppler effect over some velocity distribution. Assuming LTE to hold, such distribution will be the Maxwellian distribution. Switching into a reference frame where $\langle\vec{v}\,\rangle = 0$ (which is always possible), then the small fluctuations $\delta\vec{v}$ shall follow the Maxwellian distribution.
\par Please note that the one dimensional version of the Maxwellian distribution must be used, which is a Gaussian distribution with $\mu = 0$ and a given variance. The convoluted line profile will then be 
\begin{equation*}
    \phi(\nu) = \int \delta(\nu-\nu_{0})\nu_{0}\left(1+\frac{v}{c}\right)\exp\left(-\frac{m_{a}v^{2}}{2kT}\right)\,\text{d}v
\end{equation*}
where, as anticipated, we identified the natural broadening as a $\delta$-function. This is easily computed
\begin{equation}
    \phi(\nu) = \frac{1}{\Delta\nu_{D}\sqrt{\pi}}e^{-(\nu-\nu_{0})^{2}/(\Delta \nu_{D})^{2}} \quad \Delta\nu_{D} = \frac{\nu_{0}}{c}\sqrt{\frac{2kT}{m_{a}}}
\end{equation}
Here $\Delta \nu_{D}$ represents the \emph{Doppler width}.
\par Generally speaking, the minimum broadening we can expect is given by a convolution of Doppler Broadening and Natural Broadening in its proper Lorentzian form. This hasn't a fancy analytical solution, but can be expressed in terms of what is called a \emph{Voigt profile}.
\par Roughly speaking, the Voigt profile is something that looks like a Gaussian at the center of the line\footnote{Sometimes it's called \emph{kernel} of the distibution.} (where Doppler effect dominates) but has the 
\emph{wings} of the Lorentzian distribution.
\begin{equation*}
    H(a,u) = \frac{a}{\pi}\int_{-\infty}^{+\infty} \frac{e^{-y^{2}}}{a^{2}+(u-y)^{2}}\,\text{d}y \quad a = \frac{\Gamma}{4\pi\Delta\nu_{D}} \quad u = \frac{\nu-\nu_{0}}{\Delta\nu_{D}}
\end{equation*}
In terms of this compact definition of the Voigt profile, the overall line profile will be
\begin{equation*}
    \phi(\nu) = \frac{1}{\Delta\nu_{D}\sqrt{\pi}}H(a,u)
\end{equation*}